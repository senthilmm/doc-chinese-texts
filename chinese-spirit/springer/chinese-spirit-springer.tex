%%%%%%%%%%%%%%%%%%%% book.tex %%%%%%%%%%%%%%%%%%%%%%%%%%%%%
%
% sample root file for the chapters of your "monograph"
%
% Use this file as a template for your own input.
%
%%%%%%%%%%%%%%%% Springer-Verlag %%%%%%%%%%%%%%%%%%%%%%%%%%


% RECOMMENDED %%%%%%%%%%%%%%%%%%%%%%%%%%%%%%%%%%%%%%%%%%%%%%%%%%%
\documentclass[graybox,envcountchap,sectrefs]{svmono}

% choose options for [] as required from the list
% in the Reference Guide

\newcommand\en[1]{#1}
%\newcommand\en[1]{}
%\newcommand\zhcn[1]{#1}
\newcommand\zhcn[1]{}
%\newcommand\zhtw[1]{#1}
\newcommand\zhcna[1]{#1}

\newcommand\prologue[2]{
#1
\begin{flushright}\noindent
\hfill {\it #2}\\
\end{flushright}
\vspace{\baselineskip}
}

\newcommand\myprefaceauthor[3]{
\vspace{\baselineskip}
\begin{flushright}\noindent
#2, #3\hfill {\it #1}\\
\end{flushright}
}

\en{
  \newcommand{\doctitle}{The Spirit of the Chinese People}
  \newcommand{\docauthor}{Thomson Ku}
  \newcommand{\dockeywords}{Chinese, Spirit}
  \newcommand{\docsubject}{}
}
\zhcn{
  \newcommand{\doctitle}{春秋大義}
  \newcommand{\docauthor}{辜鴻銘}
  \newcommand{\dockeywords}{中國人, 精神}
  \newcommand{\docsubject}{中國人的精神}
}

\zhcna{

  \usepackage{ifxetex}

  \newcommand\mycjkmainfont{AR PL UMing CN}%{仿宋}%{宋體}%{新宋體}%{文鼎PL新宋}%
  \newcommand\mycjkmonofont{AR PL UMing CN}%{WenQuanYi Micro Hei Mono}
  \newcommand\mycjkboldfont{WenQuanYi Micro Hei Mono}%{AR PL UKai CN}%{YaHei Consolas Hybrid}%{黑體}%{標楷體}

  \ifxetex    % xelatex

    %% chinese setup
    \usepackage[cm-default]{fontspec} % XeLaTex 配合 fontspec 可以非常方便地設置字體。[cm-default]選項主要用來解決使用數學環境時數學符號不能正常顯示的問題
    %\usepackage{xltxtra,xunicode} %這行和上行 \usepackage[cm-default]{fontspec} 解決公式不正常的問題.但是打開後有些如 itemize 的點不能顯示。

%\setmainfont{Consolas}           % 這裏設置英文襯線字體
%\setmonofont{Consolas}           % 英文等寬字體
%\setsansfont{Consolas}           % 英文無襯線字體

    \usepackage[BoldFont, % 允許粗體
        SlantFont,        % 允許斜體
        CJKsetspaces,
        CJKchecksingle]{xeCJK}
    %\defaultfontfeatures{Mapping=tex-text} %如果沒有它,會有一些 tex 特殊字符無法正常使用,比如連字符。

    %\XeTeXlinebreaklocale "zh"                      % 重要,使得中文可以正確斷行!
    %\XeTeXlinebreakskip = 0pt plus 1pt minus 0.1pt  %

    %\setCJKmainfont[BoldFont=\mycjkboldfont]{\mycjkmainfont}
    \setCJKmainfont{\mycjkmainfont}
    %\setCJKmonofont{\mycjkmonofont}
  \fi
  %%# -*- coding: utf-8 -*-
%!TEX encoding = UTF-8 Unicode
%!TEX TS-program = xelatex
% 以上设定默认使用 XeLaTex 编译,并指定 Unicode 编码,供 TeXShop 自动识别

% 使用 LaTex 写中文的配置模板
%\documentclass[12pt]{article}

%\newcommand{\doctitle}{Peer-to-Peer Communication Across Network Address Translators}
%\newcommand{\doctitle}{穿越NAT的点对点通信}
%\newcommand{\docauthor}{Bryan Ford \& Pyda Srisuresh \& Dan Kegel}
%\newcommand{\dockeywords}{NAT穿越, 点对点, NAT Traversal, Peer-to-Peer}
%\newcommand{\docsubject}{NAT穿越}

\newcommand\mymainfont{Times New Roman}
\newcommand\mymonofont{AR PL UMing CN}%{WenQuanYi Micro Hei Mono}%{FreeMono} %{Monaco}
\newcommand\myboldfont{WenQuanYi Micro Hei Mono}%{AR PL UKai CN}%{YaHei Consolas Hybrid}%{黑体}%{標楷體}
\newcommand\mycjkmainfont{AR PL UMing CN}%{仿宋}%{宋体}%{新宋体}%{文鼎PL新宋}%
\newcommand\mycjkmonofont{AR PL UMing CN}%{WenQuanYi Micro Hei Mono}
\newcommand\mycjkboldfont{WenQuanYi Micro Hei Mono}%{AR PL UKai CN}%{YaHei Consolas Hybrid}%{黑体}%{標楷體}
\newcommand\mysansfont{FreeSans}
\newcommand\myitalicfont{Times New Roman}
\newcommand\myenglishfont{WenQuanYi Micro Hei Mono}%{FreeMono} %{Garamond}

% the algorithm2e package
\makeatletter
\newif\if@restonecol
\makeatother
\let\algorithm\relax
\let\endalgorithm\relax
\usepackage[ruled,vlined]{algorithm2e} %\usepackage[figure,ruled,vlined]{algorithm2e}

%\usepackage[margin=1.0cm,nohead]{geometry}
%\usepackage[top=1in,bottom=1in,left=1.25in,right=1.25in]{geometry} % 设置页边距
%\setlength{\belowcaptionskip}{1em} % 设置caption之后的距离

\usepackage{ifthen}
\usepackage{ifpdf}
\usepackage{ifxetex}
\usepackage{ifluatex}

\ifxetex % xelatex
\else
    %The cmap package is intended to make the PDF files generated by pdflatex "searchable and copyable" in acrobat reader and other compliant PDF viewers.
    \usepackage{cmap}%
\fi

\usepackage{url}
\usepackage{array}
\usepackage{color}
\usepackage{courier}
\usepackage{listings} % list the source code
\definecolor{ForestGreen}{rgb}{0.13,0.55,0.13}

\lstset{
    language=C,
    captionpos=b,
    tabsize=2,
    frame=lines,
    basicstyle= \normalfont\ttfamily, % \large\ttfamily, % \small\ttfamily, % \footnotesize\ttfamily, % \scriptsize\ttfamily, % Standardschrift,
    keywordstyle=\color{blue},
    commentstyle=\color{ForestGreen},
    stringstyle=\color{red},
    numbers=left,
    numberstyle=\tiny,
    numbersep=5pt,
    breaklines=true,
    showstringspaces=false,
    emph={label}
}

\definecolor{darkgreen}{cmyk}{0.7, 0, 1, 0.5}
\definecolor{darkblue}{rgb}{0.1, 0.1, 0.5}
\lstdefinelanguage{diff}
{
    keywords={+, -, \ , @@, diff, index, new},
    sensitive=false,
    morecomment=[l][""]{\ },
    morecomment=[l][\color{darkgreen}]{+},
    morecomment=[l][\color{red}]{-},
    morecomment=[l][\color{darkblue}]{@@},
    morecomment=[l][\color{darkblue}]{diff},
    morecomment=[l][\color{darkblue}]{index},
    morecomment=[l][\color{darkblue}]{new},
    morecomment=[l][\color{darkblue}]{similarity},
    morecomment=[l][\color{darkblue}]{rename},
}

\usepackage{tabularx} % long table
\usepackage{booktabs,longtable} % table in seperate pages.

% ============================================
% Check for PDFLaTeX/LaTeX
% ============================================
\newif\ifpdf
\ifx\pdfoutput\undefined
  \pdffalse % we are not running PDFLaTeX

  \usepackage[dvipdfmx,
        bookmarksnumbered, %dvipdfmx
        %% unicode, %% 不能有unicode选项,否则bookmark会是乱码
        colorlinks=true,
        urlcolor=blue,        % \href{...}{...} external (URL)
        filecolor=red,      % \href{...} local file
        linkcolor=black, % \ref{...} and \pageref{...}
        citecolor=red,
        breaklinks,
        pdftitle={\doctitle},
        pdfauthor={\docauthor},
        pdfsubject={\docsubject},
        pdfkeywords={\dockeywords},
        pdfproducer={Latex with hyperref},
        pdfcreator={pdflatex}
        %%pdfadjustspacing=1,
        pdfborder=1,
        pdfpagemode=UseNone,
        pagebackref,
        bookmarksopen=true]{hyperref}

\else
  \pdfoutput=1 % we are running PDFLaTeX
  \pdftrue

  \usepackage{thumbpdf}
  \usepackage[pdftex,
        bookmarksnumbered, %dvipdfmx
        %% unicode, %% 不能有unicode选项,否则bookmark会是乱码
        colorlinks=true,
        urlcolor=blue,        % \href{...}{...} external (URL)
        filecolor=red,      % \href{...} local file
        linkcolor=black, % \ref{...} and \pageref{...}
        citecolor=red,
        breaklinks,
        pdftitle={\doctitle},
        pdfauthor={\docauthor},
        pdfsubject={\docsubject},
        pdfkeywords={\dockeywords},
        pdfproducer={Latex with hyperref},
        pdfcreator={pdflatex}
        %%pdfadjustspacing=1,
        pdfborder=1,
        pdfpagemode=UseNone,
        pagebackref,
        bookmarksopen=true]{hyperref}

\fi

% --------------------------------------------
% Load graphicx package with pdf if needed 
% --------------------------------------------
\ifxetex    % xelatex
    %% chinese setup
    %\usepackage{xeCJK}
    \usepackage[BoldFont, % 允许粗体
        SlantFont,        % 允许斜体
        CJKsetspaces,
        CJKchecksingle]{xeCJK}
    \defaultfontfeatures{Mapping=tex-text} %如果没有它,会有一些 tex 特殊字符无法正常使用,比如连字符。

    \XeTeXlinebreaklocale "zh"                      % 重要,使得中文可以正确断行!
    \XeTeXlinebreakskip = 0pt plus 1pt minus 0.1pt  %

    \setCJKmainfont[BoldFont=\mycjkboldfont]{\mycjkmainfont}
    \setCJKmonofont{\mycjkmonofont}

    %\setmainfont{\mymainfont}        % 英文衬线字体, setmainfont=setromanfont
    %\setromanfont[Mapping=tex-text,  % 沿用 LaTex 的一些习惯的标点转换,例如 en-dash 以两个减号表示
        %Ligatures={Required,Common}, % 如果此字体内置 Ligatures 定义则启用
        %ItalicFont={\myitalicfont},  % 斜体用 Times Italic,严格来说只有拉丁子母有斜体。
        %BoldFont={\myboldfont}]      % 粗体用字体
        %{\mymainfont}                % 内文使用字体, Linux 下用 "fc-list :lang=zh-cn" 列出支持的中文字体
    %\setmonofont[Scale=0.8]{\mymonofont} % 英文等宽字体
    %\setsansfont{\mysansfont}       % 英文无衬线字体

    %\newfontfamily{\j}{Osaka}       % 设置特殊字符,这里是为日文准备的特殊字体。没有该字体,所以关闭。

    %\newfontinstance\rmfont{\myenglishfont} % 定义 rmfont 的快捷命令,可以用于在中文中显示指定的其他英文字体
    %\newcommand{\nc}[1]{{\rmfont #1}} % 如果,在中、英文夾雜時,英文想用不同的字型,則可以使用開頭時定義的 nc 指令 \nc{This is Times font.  The ‘field’ contains ligatures.}。可以比较在外面的英文字体:This is Times font.  The ‘field’ contains ligatures. 在大括号内局部设置字体:{\fontspec{Hei} (fontspec 用法和前面預設字形之 setromanfont 一樣,只是可更自由地使用)這是用日文字形的測試。只要打正常中文即可。可能會缺字就是了。}

    \usepackage[cm-default]{fontspec} % XeLaTex 配合 fontspec 可以非常方便地设置字体。[cm-default]选项主要用来解决使用数学环境时数学符号不能正常显示的问题
    %\usepackage{xltxtra,xunicode} %这行和上行 \usepackage[cm-default]{fontspec} 解决公式不正常的问题.但是打开后有些如 itemize 的点不能显示。

    \setlength{\parindent}{2.04em}  %设置首行缩进。只有中文才打开。
    \linespread{1.3}                % 设置行距

    %\definecolor{bisque}{rgb}{.996,.891,.755}
    %\pagecolor{bisque} % 设置背景颜色

    % 设置原文照排环境的字体
    % \makeatletter
    % \def\verbatim@font{\sffamily\small}
    % \makeatother

    % 将默认的英文重定义为中文
    \renewcommand{\contentsname}{目录}
    \renewcommand{\listfigurename}{插图目录}
    \renewcommand{\listtablename}{表格目录}
    %\renewcommand{\abstractname}{摘要}
    \renewcommand{\indexname}{索引}
    \renewcommand{\tablename}{表}
    \renewcommand{\figurename}{图}
    \renewcommand{\appendixname}{附录}
    %\renewcommand{\chaptername}{章节}
    %\renewcommand{\refname}{参考文献}
    %\renewcommand{\bibname}{参考}
    %\renewcommand{\IEEEkeywordsname}{关键词}

    % 设置页眉页脚
    %\usepackage[pagestyles,compact]{titlesec} % 定制页眉页脚
    %\newpagestyle{main}{%
        %\sethead[$\cdot$~\thepage~$\cdot$][][\thesection\quad%
        %\sectiontitle]{\thesection\quad\sectiontitle}{}{%
            %$\cdot$~\thepage~$\cdot$}
        %\setfoot{}{}{}\headrule}
        %\pagestyle{main}
        %\renewpagestyle{plain}{\sethead{}{}{}\setfoot{}{}{}}
    %\pagestyle{plain}

    % 设置chapter, section与subsection的格式
    %\titleformat{\chapter}{\centering\huge}{\textbf{第\thechapter{}章}}{1em}{\textbf}
    %\titleformat{\section}{\centering\LARGE}{\textbf{\thesection}}{1em}{\textbf}
    %\titleformat{\subsection}{\Large}{\textbf{\thesubsection}}{1em}{\textbf}

    % For LaN
    \newcommand{\LaN}{L{\scriptsize\hspace{-0.47em}\raisebox{0.23em}{A}}\hspace{-0.1em}N}

    % 去掉表头中的冒号
    \makeatletter
        \long\def\@makecaption#1#2{%
            \vskip\abovecaptionskip
            \sbox\@tempboxa{#1~~#2}%
            \ifdim \wd\@tempboxa >\hsize
                #1~~#2\par
            \else
                \global \@minipagefalse
                \hb@xt@\hsize{\hfil\box\@tempboxa\hfil}%
            \fi
            \vskip\belowcaptionskip}
    \makeatother

    % XeTeX logo
    \def\XeTeX{\leavevmode
        \setbox0=\hbox{X\lower.5ex\hbox{\kern-.15em\reflectbox{E}}\kern-.1667em \TeX}%
        \dp0=0pt\ht0=0pt\box0}

    \usepackage{graphicx}
\else
    \ifpdf
        \usepackage[pdftex]{graphicx}
        \pdfcompresslevel=9
    \else
        \usepackage{graphicx} % \usepackage[dvipdfm]{graphicx}
    \fi
\fi
%% \DeclareGraphicsRule{.jpg}{eps}{.bb}{}
%% \DeclareGraphicsRule{.png}{eps}{.bb}{}
\graphicspath{{./} {figures/}}
\usepackage{flafter} % 防止图形在文字前

%%%% 字体:
%Adobe Heiti Std 和 Adobe Song Std是砖头公司出的两款超pp的字体,有人把它们用在latex排版中,效果超级好。
%windows版在 Program Files/Adobe/Acrobat 8.0/Resource/CIDFont 下。

%sudo mkdir /usr/share/fonts/adobe
%sudo cp DIR2adobefonts/*.otf /usr/share/fonts/adobe
%sudo chmod 644 /usr/share/fonts/adobe/*.otf # 当前用户读写,当前组用户读写,其他用户只读

%cd /usr/share/fonts/adobe/
%sudo mkfontscale #(创建fonts.scale文件,控制字体旋转缩放)
%sudo mkfontdir #(创建fonts.dir文件,控制字体粗斜体产生)
%sudo fc-cache -fv # (建立字体缓存信息,也就是让系统认识认识)
%fc-list :lang=zh-cn # 看看装上没


%安装 LaTeX+XeTeX环境的过程, 你也使用Emacs来编辑TeX文件的话, 那么一定要安上AUCTeX这个扩展
%sudo apt-get install texlive texlive-latex-extra texlive-xetex lmodern # 首先是LaTeX与XeTeX的安装

%sudo apt-get install auctex

%安装好以后, 重点是配置.emacs文件, 因为AUCTeX本身是不支持通过xelatex编译的.
%;; AUCTeX
%(defun auctex ()
  %(add-to-list 'TeX-command-list '("XeLaTeX" "%`xelatex%(mode)%' %t; %`xelatex%(mode)%' %t" TeX-run-TeX nil t)) ;; 这里我编译了两次
    %(setq TeX-command-default "XeLaTeX") ;; 设定默认编译命令为XeLaTeX
    %(setq TeX-save-query nil)            ;; 保存之前不询问
    %(setq TeX-show-compilation t))       ;; 在新窗口显示编译过程
%(add-hook 'LaTeX-mode-hook 'auctex)

%(custom-set-variables
 %'(TeX-output-view-style (quote (("^dvi$nnnnnnn" ("^landscape$" "^pstricks$\\|^pst-\\|^psfrag$") "%(o?)dvips -t landscape %d -o && gv %f") ("^dvi$" "^pstricks$\\|^pst-\\|^psfrag$" "%(o?)dvips %d -o && gv %f") ("^dvi$" ("^a4\\(?:dutch\\|paper\\|wide\\)\\|sem-a4$" "^landscape$") "%(o?)xdvi %dS -paper a4r -s 0 %d") ("^dvi$" "^a4\\(?:dutch\\|paper\\|wide\\)\\|sem-a4$" "%(o?)xdvi %dS -paper a4 %d") ("^dvi$" ("^a5\\(?:comb\\|paper\\)$" "^landscape$") "%(o?)xdvi %dS -paper a5r -s 0 %d") ("^dvi$" "^a5\\(?:comb\\|paper\\)$" "%(o?)xdvi %dS -paper a5 %d") ("^dvi$" "^b5paper$" "%(o?)xdvi %dS -paper b5 %d") ("^dvi$" "^letterpaper$" "%(o?)xdvi %dS -paper us %d") ("^dvi$" "^legalpaper$" "%(o?)xdvi %dS -paper legal %d") ("^dvi$" "^executivepaper$" "%(o?)xdvi %dS -paper 7.25x10.5in %d") ("^dvi$" "." "%(o?)xdvi %dS %d") ("^pdf$" "." "acroread %o %(outpage)") ("^html?$" "." "netscape %o")))))

%最后那个有点长, 主要是没有找到合适的方法像添加XeLaTeX一样只需要写新增的条目, 所以这里就把原有的和修改以后的都写了出来. 其实只改了一个地方, 已经用蓝色标注出来了, 就是在使用C-c C-v预览PDF文件的时候使用什么软件来打开. 我这里就是acroread, 你用的其它的话, 可以相应修改.
%这样修改好以后, 以后就可以直接使用C-c C-c编译, C-c C-v预览, C-c `在错误间跳转了.

%但是TeX Live中的install-info文件会导致源安装AUCTeX的时候失败, 所以如果是先安装的TeX Live, 再安装AUCTeX, 就需要先把TeX Live的install-info"消灭"掉: 
%sudo mv /usr/local/bin/install-info /usr/local/bin/install-info.bak

%-------------------------------------------------------------------
%终于搞定在emacs+auctex中设置xelatex为默认编译命令!

%只要在在~/.emacs中加上

%(add-hook 'LaTeX-mode-hook (lambda()
    %(add-to-list 'TeX-command-list '("XeLaTeX" "%`xelatex%(mode)%' %t" TeX-run-TeX nil t))
    %(setq TeX-command-default "XeLaTeX")
    %(setq TeX-save-query  nil )
    %(setq TeX-show-compilation t)
    %))

%第一行参考auctex的手册auctex.pdf,版本是11.84 ;
%(add-to-list 'TeX-command-list '("XeLaTeX" "%`xelatex%(mode)%' %t" TeX-run-TeX nil t)) 会在Command 这一栏中增加了XeLaTeX这一项命令;
%(setq TeX-command-default "XeLaTeX")  则使得以后用C-c C-c就是默认用xelatex 命令编译tex文档;
%(setq TeX-save-query  nil ) 这一行不用确认保存就开始执行编译;
%(setq TeX-show-compilation t)  这一行是看到编译的滚动信息。
%现在还是在latex-mode下配置,下一步看能否在pdflatex-mode 下配置。


%(add-to-list 'TeX-command-list '("XeLaTeX" "%`xelatex%(mode)%' %t" TeX-run-TeX nil t)) 这一行中的"%`xelatex%(mode)%' %t"
%写成"xelatex  %t" 已经可以了。

}

\usepackage{mathptmx}
\usepackage{helvet}
\usepackage{courier}
%
\usepackage{type1cm}         

\usepackage{makeidx}         % allows index generation
\usepackage{graphicx}        % standard LaTeX graphics tool
                             % when including figure files
\usepackage{multicol}        % used for the two-column index
\usepackage[bottom]{footmisc}% places footnotes at page bottom

% see the list of further useful packages
% in the Reference Guide

\makeindex             % used for the subject index
                       % please use the style svind.ist with
                       % your makeindex program

%%%%%%%%%%%%%%%%%%%%%%%%%%%%%%%%%%%%%%%%%%%%%%%%%%%%%%%%%%%%%%%%%%%%%

\begin{document}

\author{\docauthor}
\title{\doctitle}
\subtitle{\docsubject}
\maketitle

\frontmatter%%%%%%%%%%%%%%%%%%%%%%%%%%%%%%%%%%%%%%%%%%%%%%%%%%%%%%

%\begin{dedication}
%\input{../chapters/chap-dedic}
%\end{dedication}

\foreword

\section[出版說明]{出版說明}

%《中國人的精神》內容簡介:
晚清以來,中國形象被嚴重扭曲。學貫中西、特立獨行的「老怪物」辜鴻銘,於1915年出版用英文寫成的《中國人的精神》,用自己的筆維護了中國文化的尊嚴,改變了部分西方人對中國的偏見。此書一出,轟動西方,後被譯為多種文字。  

在《中國人的精神》這本書中,作者把中國人和美國人、英國人、德國人、法國人進行了對比,凸顯出中國人的特征之所在:美國人博大、純樸,但不深沉;英國人深沉、純樸,卻不博大;德國人博大、深沉,而不純樸;法國人沒有德國人天然的深沉,不如美國人心胸博大和英國人心地純樸,卻擁有這三個民族所缺乏的靈敏;只有中國人全面具備了這四種優秀的精神特質。

辜湯生(1857年7月18日-1928年4月30日),字鴻銘,號立誠。祖籍福建省同安縣,生於南洋英屬馬來西亞檳榔嶼。學博中西,號稱「清末怪傑」,是中國近現代為數稀少的一位博學漢學中國傳統的同時,又精通西方語言與文化的學者。
他翻譯了中國「四書」中的三部——《論語》、《中庸》和《大學》,創獲甚巨;並著有《中國的牛津運動》(原名《清流傳》)和《中國人的精神》(原名《春秋大義》)等英文書,熱衷向西方人宣傳東方的文化和精神,產生了重大的影響,在西方形成了「到中國可以不看紫禁城,不可不看辜鴻銘」的說法。

%我們現在將英語作為一種``世界英語'' (World English) 來看待;
%於是,英語不再只是單純的一門異族語言,它同時融合者不同民族的表達方式並折射其多姿的文化。
%一個世紀以來,有過這樣的一位位中國人,他們以各自令人驚歎的完美英語,對世界解說的中國,
%對祖國表達這赤誠。
%如今,我們相信,還有更多的中國人胸懷一樣的向往,因為,跨實際的開放中國需要引進,也需要輸出。

%我們出版中國人的英語著述,正是為有志於此的英語學習者樹一個榜樣,
%為下個世紀的中國再添一份自信,還為世界英語的推廣呐一聲喊。

%選擇辜鴻銘的作品重排出版,當然不是宣揚他那不免乖張偏頗的行為思想,
%而是感動於他對中國傳統文化的奮力捍衛;驚歎於他那登峰造極、令人仰止的英語造詣。
%辜鴻銘通英法德俄等多種外語,但他的著述多用英文,而其中尤以《中國人的精神》為著。
%《中國人的精神》原載1914年的《中國評論》,1915年更名《春秋大義》在京出版,並很快被譯成德文,一時轟動西方。
%本書力闡中國傳統文化對於西方文明的價值,在當時中國文化面臨歧視、中華民族遭受欺淩的情況下,其影響尤為特殊。
%當然,對於我們現在的讀者,這首先該是一本極為寶貴的英語讀物。


\section[回憶辜鴻銘先生]{回憶辜鴻銘先生}

羅家倫

在清末民初一位以外國文字名滿海內外,而又以怪誕見稱的,那便是辜鴻銘先生了。
辜先生號湯生,福建人,因為家屬僑居海外,所以他很小就到英國去讀書,在一個著名的中學畢業,
受過很嚴格的英國文學訓練。這種學校對於拉丁文、希臘文,以及英國古典文學,都很認真而徹底地教授。
這乃是英國當時的傳統。畢業以後,他考進伯明罕大學學工程(有人誤以為他在大學學的是文學,那是錯的)。

回國以後,他的工程知識竟然沒有發揮的餘地。當時張之洞做兩湖總督,請他做英文文案。
張之洞當年提倡工業建設,辦理漢冶萍煤鐵等項工程,以「中學為體,西學為用」相號召,為好談時務之人。
他幕府裏也有外國顧問,大概不是高明的外國人士,辜先生不曾把他們放在眼裏。
有一天,一個外國顧問為起草文件,來向辜先生請問一個英文字用法。
辜默然不語,走到書架上抱了一本又大又重的英文字典,
碰然一聲丟在那外國顧問的桌上說:「你自己去查去!」這件小故事是蔡孑民先生告訴我的,
這可以看出辜先生牢騷抑鬱和看不起庸俗外國顧問的情形。

民國四年,我在上海愚園遊玩,看見愚園走廊的壁上嵌了幾塊石頭,刻著拉丁文的詩,說是辜鴻銘先生做的。
我雖然看不懂,可是心裏有種佩服的情緒,認為中國人會做拉丁文的詩,大概是一件了不得的事。
後來我到北京大學讀書,蔡先生站在學術的立場上網羅了許多很奇怪的人物。
辜先生雖然是老複辟派的人物,因為他外國文學的特長,也被聘在北大講授英國文學。
因此我接連上了3年辜先生主講的英國詩這門課程。我記得第一天他老先生拖了一條大辮子,是用紅絲線夾在頭發裏辮起來的,戴了一頂紅帽結黑緞子平頂的瓜皮帽,大搖大擺地走上漢花園北大文學院的紅樓,頗是一景。
到了教室之後,他首先對學生宣告:「我有三章約法,你們受得了的就來上我的課,受不了的就早退出:
第一章,我進來的時候你們要站起來,上完課要我先出去你們才能出去;
第二章,我問你們話和你們問我話時都得站起來;
第三章,我指定你們要背的書,你們都要背,背不出不能坐下。」
我們全班的同學都認為第一第二都容易辦到,第三卻有點困難,可是大家都J限於辜先生的大名,也就不敢提出異議。

3年之間,我們課堂裏有趣的故事多極了。
我曾開玩笑地告訴同學們說:「有沒有人想要立刻出名,若要出名,只要在辜先生上樓梯時,
把他那條大辮子剪掉,那明天中外報紙一定都會競相刊載。」當然,這個名並沒有人敢出的。
辜先生對我們講英國詩的時候,有時候對我們說:「我今天教你們外國大雅。」
有時候說:「我今天教你們外國小雅。」有時候說:「我今天教你們洋離騷。」
這「洋離騷」是什麼呢?原來是密爾頓(John Milton)的一首長詩Licidas.
為什麼Licidas會變「洋離騷」呢?
這大概因為此詩是密爾頓吊他一位在愛爾蘭海附近淹死亡友而寫成的。

在辜先生的班上,我前後背熟過幾十首英文長短的詩篇。在那時候叫我背書倒不是難事,最難的是翻譯。
他要我們翻什麼呢?要我們翻千字文,把「天地玄黃,宇宙洪荒」翻成英文,這個真比孫悟空戴緊箍咒還要痛苦。
我們翻過之後,他自己再翻。他翻的文字我早已記不清了,我現在想來,那一定也是很牽強的。
還有一天把他自己一首英文詩要我們翻成中文,當然我們班上有幾種譯文,最後他把自已的譯文寫出來了,
這個譯文是:「上馬複上馬,同我夥伴兒,男兒重意氣,從此赴戎機,劍柄執在手,別淚不沾衣,寄語越溪女,喝囑複何為!」
英文可能是很好,但譯文並不很高明,因為辜先生的中國文學是他回國後再用功研究的,雖然也有相當的造詣,卻不自然。
這也同他在黑板上寫中國字一樣,他寫中國字常常會缺一筆多一筆,而他自己毫不覺得。

我們在教室裏對辜先生是很尊重的,可是有一次,我把他氣壞了。
這是正當「五四」運動的時候,辜先生在一個日本人辦的
《華北正報》(North China Standard)裏寫了一篇文章,大罵學生運動,說我們這班學生是暴徒,是野蠻。
我看報之後受不住了,把這張報紙帶進教室,
質問辜先生道:「辜先生,你從前著的《春秋大義》(The Spirit of the Chinese People)我們讀了都很佩服,
你既然講春秋大義,你就應該知道春秋的主張是』內中國而外夷狄』的,你現在在夷狄的報紙上發表文章罵我們中國學生,是何道理?」這一下把辜先生氣得臉色發青,他很大的眼睛突出來了,一兩分鍾說不出話,
最後站起來拿手敲著講台說道:「我當年連袁世凱都不怕,我還怕你?」這件故事,現在想起來還覺很有趣味。
辜先生有一次談到在袁世凱時代他不得已擔任了袁世凱為准備帝制而設立的參政院的議員
(辜先生雖是帝制派,但他主張的帝制是滿清的帝制,不是袁世凱的帝制)。
有一天他從會場上出來,收到300銀元的出席費,他立刻拿了這大包現款到八大胡同去逛窯子。
北平當時妓院的規矩,是唱名使妓女魚貫而過,任押妓者挑選其所看上的。
辜先生到每個妓院點一次名,每個妓女給一塊大洋,到300塊大洋花完了,乃哈哈大笑,揚長而去。

當時在他們舊式社會裏,逛妓院與娶姨太太並不認為是不正當的事,所以辜先生還有一個日本籍的姨太太。
他是公開主張多妻主義的,他一個最出名的笑話就是:「人家家裏只有一個茶壺配上幾個茶杯,哪有一個茶杯配上幾個茶壺的道理?」
這個譬喻早已傳誦一時,但其本質確是一種詭辯。不料以後還有因此而連帶發生的一個引伸的譬喻。
陸小曼同徐志摩結婚以後,她怕徐志摩再同別人談戀愛,
所以對志摩說:「志摩!你不能拿辜先生茶壺的譬喻來作借口,你要知道,你不是我的茶壺,乃是我的牙刷,茶壺可以公開用的,牙刷是不能公開用的!」
作文和說理用譬喻在邏輯上是犯大忌的,因為譬喻常常用性質不同的事物作比,並在這裏面隱藏著許多遁詞。

辜先生英文寫作的特長,就是作深刻的諷刺。我在國外時,看見一本英文雜志裏有他的一篇文章,所采的體裁是歐洲中世紀基督教常用的問答傳習體(Catechism)。
其中有幾條我至今還記得很清楚,如:「什麼是天堂?天堂是在上海靜安寺路最舒適的洋房裏!
誰是傻瓜?傻瓜是任何外國人在上海不能發財的!什麼是侮辱上帝?
侮辱上帝是說赫德(Sir Roben Hart)總稅務司為中國定下的海關制度並非至善至美。」
諸如此類的問題有二三十個,用字和造句的深刻和巧妙,真是可以令人拍案叫絕。
大約是在1920年美國《紐約時報》的星期雜志上有一篇辜先生的論文,占滿第一頁全面。
中間插人一個辜先生的漫畫像,穿著前清的頂戴朝服,後面拖了一根大辮子。
這篇文章的題目是「沒有文化的美國」(The Uncivilized United States)。
他批評美國文學的盯候說美國除了Edgar Allan Poe所著的Annabelle Lee之外,沒有一首好詩。諸如此類的議論很多,可是美國這個權威的大報,卻有這種幽默感把他全文登出。
美國人倒是有種雅量,歡喜人家罵他,愈罵得痛快,他愈覺得舒服,只要你罵的技術夠巧妙。
像英國的王爾德、蕭伯納都是用這一套方法得到美國人的祟拜。
在庚子八國聯軍的時候,辜先生曾用拉丁文在歐洲發表一篇替中國說話的文章,使歐洲人士大為驚奇。
善於運用中國的觀點來批評西洋的社會和文化,能夠搔著人家的癢處,這是辜先生能夠得到西洋文藝界贊美佩服的一個理由。

無疑義的,辜先生是一個有天才的文學家,常常自己覺得懷才不遇,所以搞到恃才做物。
他因為生長在華僑社會之中,而華僑常飽受著外國人的歧視,所以他對外國人自不免取嬉笑怒罵的態度以發泄此種不平之氣。
他又生在中國混亂的社會裏,更不免憤世嫉俗。他走到舊複辟派這條路上去,亦是不免故意好奇立異,表示與眾不同。
他曾經在教室裏對我們說過:「現在中國只有兩個好人,一個是蔡元培先生,一個是我。
因為蔡先生點了翰林之後不肯做官就去革命,到現在還是革命。
我呢?自從跟張文襄(之洞)做了前清的官以後,到現在還是保皇。」這可能亦是他自己的解嘲和答客難吧!



\preface

The object of this book is an attempt to interpret the spirit and show the value of the Chinese civilisation.
Now in order to estimate the value of a civilisation, it seems to me, the question we must finally ask is not what great cities, what magnificent houses, what fine roads it has built and is able to build;
what beautiful and comfortable furniture,
what clever and useful implements, tools and instruments it has made and is able to make;
no, not even what institutions, what arts and sciences it has invented:
the question we must ask, in order to estimate the value of a civilisation,
-- is, \emph{what type of humanity},
what kind of men and women it has been able to produce.
In fact, the man and woman,
-- the type of human beings
-- which a civilisation produces,
it is this which shows the essence, the personality,
so to speak, the soul of that civilisation.
Now if the men and women of a civilisation show the essence,
the personality and soul of that cifilisation, the language which a man and woman speak, shows the essence, the personality, the soul of the man and woman.
The French say of literary composition, ``\emph{Le style, c'est l'homme.}''
I have therefore taken these three things, the Real Chinaman, the Chinese woman and the Chinese language,
-- as the subjects of the first three essays in this volume to illustrate the spirit
and show the value of the Chinese civilisation.

I have added to these, two essays in which I have tried to show how and why men, foreigners who are looked upon as authorities on the subject, do not really understand the real Chinaman and the Chinese language. The Rev. Arthur Smith, who wrote the Chinese Characteristics, I have tried to show, does not understand the real Chinaman, because, being an American, --he is not deep enough to understand the real Chinaman. Dr. Giles again, who is considered a great sinologue,
I have tried to show, does not really understand the Chinese language, because, being an Englishman, he is not broad enough, -- he has not the philosophic insight and the broadness which that insight gives, I have wanted to include in this volume and essay I wrote on J. B. Bland and Backhouse's book on the famous late Empress Dowager, but unfortunately I have not been able to find a copy of that essay which was published in the ``National Review'' in Shanghai some four years ago.
In that essay, I have tried to show that, such men as J.B. Bland and Backhouse do not and cannot understand the real Chines woman, -- the highest type of woman produced by the Chinese civilisation viz the late Empress Dowager, because such men as J. B. Bland and Backhouse are not simple, -- have not the simplicity of mind, being too clever and having, like all modern men,
a distorted intellect. \footnote{Mencius says, ``What I hate in your clever men is that they always distort things. 所惡於智者為其鑿也'' Bk IV. Part II. 26.}
In fact, in order to understand the real Chinaman and the Chinese civilisation, a man must be deep, broad and simple, for the three characteristics of the Chinese character and the Chinese civilisation are: depth, broadness and simplicity.

The American people, I may be permitted to say here,
find it difficult to understand the real Chinaman and the Chinese civilisation,
because the American people, as a rule, are broad, simple, but not broad.
The Germans again cannot understand the real Chinaman and the Chinese civilisation because the Germans, especially the educated Germans, as a rule, are deep, broad, but not simple.
The French, -- well the French are the people, it seems to me, who can understand 
and has understood the real Chinaman and the Chinese civilisation best.
\footnote{The best book written in any European Language on the spirit of the Chinese civilisation is a book
called ``La Cit\'e Chinoise'' by G. -- Eug. Simon who was once French Consul in China.
It was from this book that Prof. Lowes Dickinson of Cambridge, as he himself told me,
drew his inspiration in writing his famous ``Letters from John Chinaman.''}
The French, it is true, have not the depth of nature of the Germans nor the broadness of mind of the Americans nor the simplicity of mind of the English,
-- but the French, the French people have to a preeminent degree a quality of mind
such as all the people I have mentioned above as a rule, have not,
-- a quality of mind which, above all things, is necessary in order to understand the real Chinaman and the Chinese civilisation;
a quality of mind viz: \emph{delicacy}.
For, in addition to the three characteristics of the real Chinaman and Chinese civilisation
which I have already mentioned,
I must here add one more, and that the chief characteristic, namely delicacy;
delicacy to a preeminent degree such as you will find nowhere else except perhaps among the ancient Greeks and their civilisation.

It will be seen from what I have said above that the American people if they will study the Chinese civilisation, will get depth;
the English, broadness; and Germans, simplicity;
and all of them, Americans, English and Germans by the study of the Chinese civilisation, of Chinese books and literature, will get a quality of mind which,
I take the liberty of saying here that it seems to me, they all of them, as a rule,
have not to a preeminent degree, namely, \emph{delicacy}.
The French people finally, by the study of the Chinese civilisation,
will get all, -- depth, broadness, simplicity and a still finer delicacy than the delicacy which they now have.
Thus the study of the Chiness civilisation, of Chinese books and literature will,
I believe, be of benefit to all the people of Europe and America.
I have therefore added to this volumen an essay on Chinese scholarship, -- the sketch of a programme how to study Chinese, which I made for myself when I made up my mind and began,
after my return from Europe, to study the civilisation of my own country,
exactly now thirty years ago;
this sketch of a programme how to study Chinese and the Chinese civilisation.

Last of all, I have included as an appendex an essay on pratical politics,
-- an essay on ``The War and the Way out.''
Knowing full well the danger of entering into the arena of pratical politics,
    I nevertheless do it, because in order to prove the value of the Chinese civilisation, I want to show how the study of the Chinese civilisation can help to solve the problem facing the world to-day, -- the problem of saving the civilisation of Europe from bankruptcy.
In fact I want to show that the study of Chinese, of Chinese books and Chinese literature is not only a hobby for sinologues.

In this essay, I have tried to show the moral causes which have brought on this war;
for until the ture moral causes of this war are understood and remedied,
there can be no hope of finding a way out of it.
The moral causes of this war, I have tried in my essay to show, are the \emph{worship of the mob} in Great Britain and the \emph{worship of might} in Germany.
I have, in my essay, laid emphasis more upon the worship of the mob in Great Britain, than the worship of might in Germany,
because looking impartially upon the question, it seems to me that it is the worship of the mob in Great Britain, which is responsible for the worship of might in Germany;
in fact, the worship of the mob in all European countries and especially in Great Britain,
it was this which has created the enormous German Millitarism which everybody now hates and denounces.

Now let me first of all say here that it is the \emph{moral fibre} in the German nation, their intense love of righteousness and, as a consequence, their equally intense hatred of unrighteousness,
hatred of all untidiness and disorder (Unzucht und Unordnung),
which makes the German people believe in and worship might.
All men who intensely love righteousness,
who intensely hate unrighteousness are inclined to believe in and worship might.
Why? Because Carlyle with the German moral fibre in him intensely hated unrighteousness.
Now the reason why I say that it is the worship of th emob in Great Britain
which is responsible for the worship of might in Germany,
is because, the \emph{moral fibre} -- the intense hatred of unrighteousness,
of untidiness and disorder in the German nation makes them hate the mob,
the worship of the mob and the mob worshippers in Great Britain.
After the German nation saw how the mnob and the mob-worshipping politicians of Great Britain made the Boer War in Africa,
their \emph{instinctive} intense hatred \footnote{The famous telegram of the German Emperor to President Kruger was an \emph{instinctive} outsburst of indignation of the true Gernamic soul with its moral fibre against Joseph Chamberlain and his Cockney class in England, who manipulated the Boer War.}
for the mob, the mob-worship and the mob-wroshippers in Great Britain made the German nation willing to make heavy sacrifices, make the \emph{whole German nation ready to starve themselves to create a Navy} with the hope to put downn the mob,
the mob-worship and mob-worshippers in Great Britain.
In fact, the German nation, I may say, found themselves surrounded on all sides by the mob, mob-worship and mob-worshippers encouraged by Great Britain in all Europe and this made the German nation believe more and more in might,
made the German nation worship might as the only salvation for mankind.
This worship of might in Germany created by the hatred for the Religiion of mob worship in Great Britain, thus created the enormous monstrous German Militarism which everybody now hates and denounces.

Thus, I say again, it is the worship of the mob, the Religion of the worship of the mob in all European countries, especially in Great Britain, which is responsible for the worship of might in Germany;
which has created the abnormous enormity and monstrosity of German Militarism in Europe to-day.
If therefore the people in Greate Britain and the people in all European countries and America want to put down German Militarism, -- they must try first to put down the mob, the mob-worshippers and the Religion of mob-worship in their own countries.
\footnote{Confucius said to a disciple ``when outside nations are dissatisfied with you, you should cultivate \emph{civil or Civic} virtues (遠人不服則修文德).''
The Brithish aristocracy, however, like the Manchu aristocracy in China, are now helpless against the mob and mob worshippers in England.
But it is, I must say, a great credit to the British aristocracy that not one of them as far as I know, has joined the mob in England in their shout, howl and yell in this war.}
To the people of Europe and America, and in Japan and China too, to-day who speak of and want liberty, I will venture here to say that the only way, it seems to me, to get liberty, true liberty is to behave themselves;
to learn to behave themselves properly.
Look at China before this Revolution.
There was more liberty among the Chinese people,
-- no priest, no policeman, no Municipal tax,
no income tax to bother them -- more liberty among the Chinese than among any other people in the world;
and why? Because the Chinese people before this Revolution behaved themselves;
knew how to behave themselves;
knew how to behave themselves as \emph{good citizens}.
But now after this Revolution -- there is less liberty in China, and why?
Because the modern queueless, up-to-date Chinamen,
the returned students have learnt from the people of Europe and America,
-- learnt from the European mob in Shanghai how to \emph{misbehave themselves};
to behave themselves not as good citizens,
but as a \emph{mob} -- a mob encouraged, coddled and worshipped by the British diplomats and the British Inspector General of Customs in Peking.
\footnote{To show what a mob the Chinese returned students have became, i may mention here that some of these students in Peking last year actually wrote letters to the ``Peking Gazette,'' a newspaper conducted by a clever Chinese ``Babu'' by the name of Eugene Chen, openly threatening to organise and carry out a public assault upon me for criticising the new Chinese woman in my essay on ``the Chinese worman.'' This clever Chinese ``Babu'' Eugene Chen the instigator of the contemplated piece of rowdyism now is a respected member of the Committee of the Anglo-Chinese Friendship Bureau under the patronage of the British Minister and the I. G. of the Chinse Customs!}
In fact, what I want to say here is, that if the people in Europe, the people in Great Britain want to put down German Militarism, Prussian Militarism,
they must keep the mob in their own countries in order;
they must make the mob in their own countries behave themselves properly;
in fact they must put down the Religion of mob-worship,
and the mob-worshippers in their own countries.

But now while I say that the British people with their mob worship and encouragement of mob-worship are responsible for the worship of might in Germany,
for German Militarism, I must at the same time say here that,
looking again impartially upon the question, it seems to me that the \emph{direct}
responsibility for this war nests more heavily upon the German people,
upon the German nation, than upon anybody else.

In order to understand this, let the first of all here give the history of German Militarism in Europe.
After the Reformation and the Thirty Years War,
the Germanic nations, the people of the Germanic race with their \emph{moral fibre},
with their intense love of righteousness and their intense hatred of unrighteousness,
hatred of all untidiness and disorder, the German people, with Militarism as a sword in their hand,
became the rightful guardian of civilisation in Europe.
In other words, the responsibility for putting order and tidiness (Zucht und Ordnung) in Europe;
in fact, the \emph{moral hegemony} so to speak of Europe came into the hands of the German people.
After the Reformation, Frederick the Great, like Cromwell in England, had to take up and use the sword of German Militarism to put order and tidiness in Europe and he succeeded in putting order and tidiness at any rate in the Northern part of Europe.
Now see what happened after Frederick the Great's death.
His successor did not know how to use the sword of German militarism in order to guard and protect the civilisation of Europe;
in fact, he was unfit to hold the moral hegemony of Europe.
The result was, the whole of Europe, even the courts in Germany became a bottomless pit of abominations covered up only with the veneer of civilisation;
so much so that at last the suffering population, the plain men and women in France rose up with pikes to protest against the abominations.
The plain men and women in France who rose up to protest against the abominations very soon became a \emph{mob}, and this mob finally found a great and able leader,
Napoleon Bonaparte, \footnote{Emerson with great insight, says, ``What sent Napoleon to St. Helena, was not loss of battles, but the \emph{parvenu}, the vulgar ambition in him -- the vulgar ambition to marry a real Princess, to found a dynasty.''}
who led them to rob, murder, kill and ravage all Europe until the nations in Europe rallying round the small nucleus of \emph{sound} German Militarism left in Europe,
put an end to the career of the great leader of the mob at Waterloo.
After this the mnoral hegemony of Europe should have returned to the people of the Germanic race, to the Prussians, the back bone of the German nations.
But the jealousy of the other races which formed the Austrain Empire prevented this.
The result was that without the German nation with its moral fibre and the sword of German Militarism to keep down the mob, the mob in 1848 again rose up furiously to break the civilisation, of Europe.
Then again the German nation -- the backbone of the Germanic nations,
the Prussians with their moral fibre and the sword of German Militarism, saved Europe, -- saved Kingship, (Bismarck called it the dynasty), saved civilisation in Europe from the mob.

But now the Austrians, -- the other races forming the Austrian Empire again became jealous and would not allow the German nation, -- the backbone of the Germanic nation, Prussia to take ove the moral hegemony of Europe until 1866 when the Prussian King Wilhelm with Bismarck and Moltke had to put down the Austrian jealousy by force and took over the hegemony into their hands.
After this, Louis Napolean, not like his great uncle a leader, but a swindler of the mob or, as Emerson calls him, a successful thief, tried with the mob of Paris behind him, to dispute and wrest the moral hegemony of Europe from the German nation.
The result was that the Emperor Wilhelm with the sharp sword of German militarism in his hand had to march to Sedan and put down the poor successful thief and swindler of the mob.
The plain men and women of Paris who put their trust in the mob and the swindler of the mob had their houses sacked and burnt \emph{not by the German Militarism}, not by the Germans and Prussians, but by the very mob in whom they put their trust.
After 1872, -- not only the moral, but the actual political hegemony of Europe passed finally into the hands of the German nation with the moral fibre of the Germanic race in their soul and the sword of German Militarism in their hand,
to hold down the mob and keep the peace in Europe and thanks to the moral fibre in the German nation and the sword of German Militarism, Europe since 1872 has now enjoyed peace for 43 years.
Thus people who abuse and denounce German Militarism and Prussian Militarism should remember how much Europe owes to this very German, this Prussian Militarism which they now abuse and denounce.

I have in the above taken the trouble to give this rough short sketch of the German Militarism in Europe in order to make the German people see that I am not prejudiced against them in saying what I am going to say to show that the actual \emph{direct} responsibility for this war rests more heavily upon them,
upon the German people and German nation than upon anybody else.
I say that the actual \emph{direct} responsibility for this war rests more heavily upon the German people and German nation than upon anybody else;
and why? -- Because \emph{power means responsibility}.\
\footnote{Confucius says, ``Possession of power without leniency and generosity is a thing which I never can bear to see. (居上不寬吾何以觀之) Shakespeare says: ``Oh, it is glorios to have a giant's strength: but it is tyrannous to use it like a giant.''}

I say that it is the intense love of righteousness, the intense hatred of unrighteousness, intense hatred of all untidiness and disorder (Unzucht und Unordnung) in the German people which makes them believe in and worship might.
Now I want to say here that this hatred of unrighteousness, hatred of untidiness and disorder, when it becomes over-intense, when it is carried to excess becomes also an \emph{unrighteousness}, becomes a frightful and terrible unrighteousness, something more sinful and wrong even than untidiness and disorder.
It was this over intense hatred of unrighteousness which came from their intense love of righteousness, the intese, narrow, hard, rigid hatred of unrighteousness carried to excess in the old Hebrew people -- the Hebrew people to whom the people of Europe owe their knowledge and love of righteousness, it was this which destroyed the Jewish nation.
It was from this over-intense narrow, hard, rigid hatred of unrighteousness that Jesus Christ came to save Hist people.
Christ, with what Matthew Arnold calls his unspeakable sweet reasonableness said to his own people: ``Learn of me, that I am \emph{mild} and lowly and yet shall have peace in your souls.'' But the Jews -- his own people would not listen to him;
they, instead of listening to him, crucified him and the consequece was --
the Jewish nation perished.
To the Romans who were then the guardians of civilisation in Europe, Christ said, ``All they that take the sword shall perish with the sword!''
\footnote{That is to say, all who depend and put their faith solely upon material brute force or as Emerson says, who believe in the vulgar musket worship.}
But the Romans would not listen, but instead, all owed the Jews to crucify him.
The consquence was -- the Roman Empire and the old civilisation of Europe perished and passed away.
Goethe says: ``What a long way mankind must have travelled before they came to know how to deal gently even with sinners, to be merciful to law-breakers,
\emph{and to be human even to the inhuman}.
Truly they were men of Divine nature who first taught this and who gave their lives for it in order to make the realisation of this possible and to hasten the practice of it.
``(Welchen Weg musste nicht die Menschheit machen, bis sie dahin gelangte, auch gegen Seluldige gelind, gegen Verbrecher schonend, gegen auch Unmenschliche menschlich zu sein. Gewiss waren as Manner g\"ottlicher Natur, die dies zuerst lehrten, die ihr Leben damit zubrachten, die Aus\"ubung m\"oglich zu machen und zu beschleunigen.)''

With those words of their great Goethe I will endure here to appeal to the German people, to the Germann nation and say to them that,
unless they find a way to put down their narrow, hard, rigid, excessive hatred of unrighteousness which makes them believe so absolutely in and worship might,
unless they put away their absolute belief in and worship of might -- they, the German nation, like the Jewish nation, will perish and what is more,
the modern civilisation of Europe for want of a strong guardian, will collapse and pass away just as the ancient civilisation of Europe passed away.
For it is this overintense, narrow hard, rigid hatred of unrighteousness which makes the German people, the German nation believe in and worship might;
and it is this absolute belief in and worship of might which makes the German nation, the German diplomats German officials and the German people so inconsiderate and tactless in their behaviour towards other people.
When my German friends have asked me to show them a proof of the German worship of might, of German tactlessness, I have simply pointed the Kettler memorial in Peking to them.
The Kettler memorial in Peking is a standing monument of the German worship of might, of the tactlessness of the German diplomacy,
the tactlessness of the German nation in their international dealings with other nations.
\footnote{The German Minister Baron Kettler during the fanatic Boxer outbreak in China was accidentally killed by a madman from the fanatic soldiery.
As a punishment for this act of a mad man, the German diplomats insisted upon brandding the whole Chinese nation on the forehead with an indelible mark of humiliation, by having this kettler memorial erected in the principal street of the Chinese Capital.
See note on Page 12. The later Count Cassini, Russian Minister in Peking just before the Boxer outbreak, said in an interview with an American journalist,
``The Chinese are a \emph{polite people}, but the impoliteness of the British and German Ministers, -- especially of the German Minister in Peking is something outrageous.''
}
It was this worship of might of the German nation, this tactlessness of the German diplomacy of which the Kettler memorial is a standing monument, which made the Emperor of Russia say: ``We have stood this for seven years; now it must finish;''
this tactlessness of the German diplomacy which made the really peaceloving Emperor of Russia and the best people in Europe,
the soundest, most loveable, kindest and most generous-hearted people in Europe the Russians take the side of the mob and mob-worshippers in Great Britain and in France, which created the Triple-Entente;
which made the Russians finally take the side even of the anarchic mob in Servia and thus brought on this war.
In one word it is this tactlessness of the German diplomacy, of the German people, of the German nation which is \emph{directly} reponsible for this war.

I say therefore, if the German nation at this moment the true,
rightful and legitimate grardian of the modern civilisation of Europe to-day,
is not to perish and the mnodern European civilisation is to be saved,
-- the German nation, the German people must find a way to put down their over-intense, narrow, hard, rigid hatred of unrighteousness which makes them believe so absolutely in and worship might;
in fact they must find a way to put down their absolute belief in and worship of might which makes them so inconsiderate and tactless.
But then, where are the German nation, the German people to find a way to put down their absolute belief in and worship of might?
The German nation, the German people, I say, will find this in these words of their great Goethe.
Goethe says: ``\emph{There are two peaceful powers in this world: Right and Tact.}'' (Es gibt zwei friedliche Gewalten auf der Welt: Das Recht und die Schicklichkeit.)

Now this Right and Tact, \emph{das Recht und die Schicklichkeit},
is the essence of the Religion of good citizenship which Confucius gave to us Chinese here in China;
this Tact, this \emph{Schicklichkeit}, especially, is the essence of the Chinese civilisation.
The Religion in the civilisation of Right, but it did not teach Tact.
The civilisation of Greece taught the people of Europe the knowledge of Tact but it did not teach Right.
But the Religion in the civilisation of China teaches us Chinese both Right and Tact, -- das recht und die Schicklichkeit.
The Hebrew Bible, the plan of civilisation according to which the people of Europe to love righteousness, to be righteous men, to do right.
But the Chinese Bible -- the Five Canons and Four Books in China,
the plan of civilisation which Confucius saved for us the Chinese nation, teaches us Chinese also to love righteousness;
to be righteous men; to do right, but it adds: ``Love righteousness, be righteous men, do right -- but \emph{with good taste}.''
In short, Religion in Europe says: ``Be a good man.''
But the Religion in China says: ``Be a good man \emph{with good taste}.''
Christianity says: ``Love Mankind.''
But Confucius says: ``Love Mankind \emph{with good taste}.''
This Religion of righteousness with good taste, which I have called the Religion of good citizenship, is the new religion I belive,
which the people of Europe, especially the people of the countries now at war,
want at this moment not only to put an end to this war,
but to save the civilisation of Europe,
to save the civilisation of the world.
This new Religion, the people of Europe will find here in China,
-- in the Chinese civilisation.
I have therefore in this little book made the attempt to interpret and show the value of this,
-- the Chinese civilisation.
I do this with the hope that all educated serious thinking people,
who read this book of mine will, by reading this book, better understand the moral causes of this war and understanding the moral causes of this war,
will all help to put an end to this cruel, inhuman, useless and most monstrous war which the world has ever seen.

Now if we want to help to put an end to this war,
we must, all of us, try to put down first the 
worship of the mob and then the worship of might 
in the world to-day, which, as I have said, are the 
cause of this war. We can put down the worship of 
the mob, only when in our daily life, in everything 
we say and do, every one of us will think, not of 
interests, of expediency -- \emph{of what will pay}, but think 
of that word in Goethe's saying -- \emph{Right}. Confucius 
says: ``The gentleman understands \emph{right}; the \emph{cad}
understands \emph{interests}, -- \emph{what will pay}.''
Further we can only put down the worship of the mob in the world 
when we have the courage, even if it does not pay to 
do so, to refuse to join and go in with the crowd -- with 
the \emph{mob}. Voltaire says: ``C'est le malheur des gens 
honn\^etes qu'ils sont des laches.
It is the mirfortune of so-called good people that they are cowards.''
For it is the selfishness and cowardice in all of us,
I want to say here, selfishness which makes us think of 
interests, of expediency, of what will pay, instead of 
right, and cowardice which makes us afraid to stand 
up alone against the crowd, against the mob,
-- it is this selfishness and cowardice in all of us which has given 
rise and created the mob and the worship of the mob 
in the world to-day. People say German Militarism 
is the enemy and danger of the world to-day.
But I say it is the selfishness and cowardice in all of us 
which is the real enemy of the world to-day : selfishness 
and cowardice in all of us, which, when combined, 
becomes Commercialism.
It is this spirit of Commercialism, in all countries of the world,
especially in Great Britain and America,
which is 
the real enemy of the world to-day.
It is, I say, this spirit of Commercialism in all of us and not 
Prussian Militarism which is the real, the greatest 
enemy of the world to-day. For it is this Commercialism, 
a combination of selfishness and cowardice which has 
created the Religion of the worship of the mob 
and it is this Regigion of the worship of the mob 
in Great Britain which has created the Religion 
of the worship of might in Germany, created the 
German Militarism which, as I said, firrally brought 
on this war. \emph{The fons et oriyo} of this war, I say, 
therefore is not militarism, but \emph{Commercialism}, which, 
as I said, is a combination of selfishness and 
cowardice in all of us.
Thus, if we want to help to put an end to this war,
we must, all of us, first put down the spirit of Commercialism,
the combination of selfishness and cowardice in us;
in short, we must first of all, think of \emph{right} and not of interests and 
then have the courage to stand up against the crowd, 
against the mob.
In this way, I say, and \emph{only in this way} we can help to put down the worship of 
the mob, the Religion of the worship of the mob and 
in putting down this worship of the mob, this 
Religion of the worship of the mob, we can help to 
put an end to this war. 

Now as soon as we have put down the worship of 
the mob, it will then be very easy to put down the 
worship of might, easy to put down German Militarism, 
put down Prussian Militarism.
The only thing we will have to do, in order to put down the worship 
of might, to put down German, Prussian or any 
Militarism in the world, is to think of the other 
word in that saying of Goethe -- \emph{Schicklichkeit, Tact, 
Good Taste} and, in thinking of that, to behave 
with tact and good taste, in short to behave 
properly;
for might, Militarism, even Prussian 
Militarism can do nothing and will soon find itself 
useless and unnescessary against people who know 
how to behave themselves properly.
This then is the 
essence of the Religion of good citizenship; this is the 
secret of the Chinese civilisation. This is also the 
secret of the new civilisation of Europe which the 
German Goethe gave to the people of Europe and the 
secret of this civilisation is: to put down force, \emph{not} by 
force, but by \emph{right and tact};
in fact to put down force 
and everything that is evil in this world, not by force, 
but by ordering our conversation aright and behaving
ourselves properly;
and ordering our conversation
aright and behaving properly means \emph{to do right and to
behave with tact and good taste}. 
\footnote{Confucius says, ``The moral man, the gentleman by living a life 
of simple truth and earnestness can bring peace to the world (君子篤恭而天下平).}
This is the secret, the 
soul of the Chinese civilisation, the essence of the 
spirit of the Chinese people, which I have tried in this 
book to interpret and explain. 

Finally I will here conclude with the words with 
which I concluded the book ``Papers from a Viceroy's 
Yamen'' which I wrote after the Boxer trouble in China.
They are the words of the French poet 
B\'eranger and I think they are very appropriate at 
the present moment.

\begin{quote}
J'ai vu la Paix descendre sur la terre,

Semant de l'or des fleurs et des \'epis;

L' air \'etait calme et du Dieu de la guerre

Elle \'etouffait les foudres assoupis.

Ah! disait-elle, egaux par la vaillance.

Anglais, Francais, Belge, Russe ou Germain,

Peuples, formez une sainte alliance

Et donnez vous la main!
\footnote{我目睹和平徐徐降臨,

她把金色的花朵麥穗灑遍大地:

戰爭的硝煙已經散盡,

她抑制了使人昏厥的戰爭霹靂。

啊!她說,同樣都是好漢,

英法比俄德人

去結成一個神聖同盟

拉起你的手吧!

-- 據黃興濤譯文,下同。編者注。
}
\end{quote}

\myprefaceauthor{Ku Hung-Ming}{Peking}{20th April, 1915}


%\extrachap{Acknowledgements}
%\include{chapters/chap-acknow}

\tableofcontents

%\extrachap{Acronyms}
%\include{acronym}
%Use the template \emph{acronym.tex} together with the Springer document class SVMono (monograph-type books) or SVMult (edited books) to style your list(s) of abbreviations or symbols in the Springer layout.

%Lists of abbreviations\index{acronyms, list of}, symbols\index{symbols, list of} and the like are easily formatted with the help of the Springer-enhanced \verb|description| environment.

%\begin{description}[CABR]
%\item[ABC]{Spelled-out abbreviation and definition}
%\item[BABI]{Spelled-out abbreviation and definition}
%\item[CABR]{Spelled-out abbreviation and definition}
%\end{description}

\mainmatter%%%%%%%%%%%%%%%%%%%%%%%%%%%%%%%%%%%%%%%%%%%%%%%%%%%%%%%
%\begin{partbacktext}
%\part{Part Title}
%\input{part}
%\noindent Use the template \emph{part.tex} together with the Springer document class SVMono (monograph-type books) or SVMult (edited books) to style your part title page and, if desired, a short introductory text (maximum one page) on its verso page in the Springer layout.
%\end{partbacktext}

\input{../chapters/chap-intro}

\input{../chapters/chap-spirit}

\input{../chapters/chap-woman}

\input{../chapters/chap-lang}

\input{../chapters/chap-john}

\input{../chapters/chap-sinologue}

\input{../chapters/chap-scholarship1}

\input{../chapters/chap-scholarship2}


\appendix
\input{../chapters/chap-apnd1}


\chapter{Uncivilized United States}

\begin{quotation}
The following article was originally published in the North China Standard of Peking, China, and is furnished to {The New York Times} by the author. 
\end{quotation}

An American,
who was irritated because in an article I wrote lately I had said that the English were the only modern people in the world today who knew how to govern an empire,
said to me,
``what about us Americans?
Although the United States is called a Republic,
yet do we not govern a country as big as the British Empire?''
``Yes,'' I said in reply, ``but there is one great difference between the British people and you Americans.
The British people as a nation is a nation with a civilization,
whereas you Americans, living now in your plaster and concrete sky-scraping tents,
are still a nomad nation without any civilization.''
``oh,'' said the American then to me, angrily and with a sneer,
`` you say that because you have been educated in England and have never been in America!'' 

Indeed. I now remember how a Chinese Minister to America,
the late Chang Yin-huan, who was executed during the Boxer trouble,
once created great astonishment as well as amusement among the Americans in America when he told them that he found everything in America which a man wants except -- religion.
``what,'' said an American newspaper at the time,
``we Americans have no religion -- we, with a church in every street and our missionary societies!
Why, we in America have so much religion that we can afford to export to China and Korea more of that article than any other country!''

Nevertheless, the Chinese Minister, I must say, was quite right in what he said.
Only what he really wanted to say was not that there was no religion,
but that there was no civilization in America.
The real Chinese word for civilization, \emph{li-yo}
(literally ceremony or forms of courtesy and music),
rightly translated by the early Jesuit missionaries as religion,
which the Chinese Minister had in his mind, means both religion and civilization. For, not as in Europe and America today -- where religion is one thing, a something for use only on Sundays, and civilization is quite another thing, a something for use on the other six days of the week -- in China, religion is civilization and civilization is religion; they both mean one and the same thing, namely, form or expression of spiritual life which is for use not only on Sundays, but on every day of a man's life.

But before going further, let me here explain what I mean by a nation with a civilization and a nation without any civilization.
Now we all say that the ancient Greeks and the ancient Romans were great civilized nations.
Why? Because, beside governing and fighting,
besides producing material goods and making money by selling them,
these ancient nations also produced spiritual things such as art and literature and,
what is still more important,
by their art and literature they developed high, perfect types of humanity in their great men
-- all which, now after they have perished and ceased to be nations,
are remembered, admired and prized by men of all after generations.
In short, a civilized nation is a nation which has a spiritual asset or,
as Carlyle calls it. ``realized ideals.'' 

The reason, therefore, why I have said that the British people as a nation is a nation with a civilization is because, besides shopkeeping, winning Waterloo battles and governing an Indian Empire, the British nation, like the ancient Greeks and Romans, has also produced one very great spiritual thing
-- with perhaps one exception, the greatest thing, in my opinion,
which modern Europe has yet produced
-- and that is William Shakespeare.
Speaking of Shakespeare. Carlyle,
in his ``Heroes and Hero Worship ,'' rightly says,
``He is the grandest thing which we British people have yet done.''
If the great British Empire were destroyed tomorrow, a thousand years after this, when men read the works of Shakespeare,
they would say that the British nation was a nation with a very high civilization.

Indeed, as the one Latin word ``virtus'' -- not virtue in the English sense, but the virtue of the Japanese Samurai -- is a proof to those who understand Latin that the ancient Romans were a nation with a very noble civilization
-- so the English word ``gentleman'' alone, without any Shakespeare, is enough to show that the British nation is nation with even a finer civilization than the noble civilization of the ancient Romans, because it is a civilization which, tempered with the spirit and ideal of gentleness of Christianity,
has produced a type of humanity called a ``gentleman.''
For the chief and one aim of civilization is not to make and teach men to be strong,
but to make and teach men to be gentle;
in other words, to develop and produce not what Kipling calls coarse, vulgar, ``flannelled fools'' who can yell as the American Y. M. C. A. in China now trying to do
-- but to develop and produce gentlemen,
who, as we Chinese say, understand li-yo, courtesy, good manners or ``good form,'' as the Englishman calls it, and music.

Indeed, I may say here, it is because the ideal of the British or English civilization is to make a ``gentleman'' that the British people are the only modern people in the world today,
as I have said, who can govern an Empire. The great Soldier-Gentleman of Japan, Tokugawa lyeyasy,
after he had, with his sharp sword, cast the ``devil of cruelty'' out of old feudal Japan
-- just in the same way as the British ``Unknow Warrior.''
Whom they lately buried in Westminster Abbey in England,
has now cast the ``devil'' called Furor Teutonicus out of feudal Germany
-- was on his death-bed, he sent for his grandson lyemistsu and said to him,
``You are the man who one day will have to govern an empire.
Remember, the way to govern an empire is to have a gentle and tender heart
(the Latin alma as in alma mater, the extreme gentle tenderness of a mother).''

Now the reason, it seems to me, why the Japanese statesmen now find it so difficult to govern Korea,
is because the modern Japanese,
instead of reading and teaching their students the Guai Shi,
now read and teach them the pragmatic philosophy and political science of Professor Deway,
and have thus forgotten the essence of political science contained in those words of their great Shogun which I have just quoted.

In the same way, the reason why the British politicians now in England find it difficult to govern Ireland and India,
is because modern Englishmen today do not know that it was not British democracy,
the British Constitution or Parliamentarism,
but the British or English civilization with the ``gentleman'' and its ideal;
in short, that it was not the British mob, but the British or English gentleman who built up the great British Empire of today. But that is neither here nor there.

I have said that a nation is called a civilized nation only when it has a spiritual asset or ``realized ideals.''
Now let me ask, what ``realized ideals'' or spiritual asset have the Americans today to show in order to entitle them to be called a civilized nation?
In literature I know only one great name in America and that is Emerson.
But then even Emerson, as Matthew Arnold says, is not quite a great name in literature.
Without speaking of Homer, Virgil Dante and Shakespeare, Emerson is not a great name in literature even as Plato, Cicero Bacon and Voltaire are great names in literature.

Of poetry again, which, like music, is the highest expression of spiritual life in a nation,
I also know only one poem written by an American poet which can be truly called a real poem.
By a real poem I mean a poem which is all poetry and nothing but poetry;
a poem which becomes the spiritual asset of a nation and forms an important part of its civilization,
such as Gray's ``Elegy in a Country Churchyard'' and Robert Burns's ``Auld Lang Syne.''
Poems like the poems of the English Lord Macaulay are, although in meter and rhyme,
not poetry at all, but only rhetoric.
Now, the poems of even famous American poets like Longfellow and John Greenleaf Whittier are for the most part also rhetoric,
with some poetry in them: they are not all poetry,
not real poems, which like Robert Burns's ``Auld Lang Syne,'' can become the spiritual as set of a nation.
Indeed, as I have said, the only poem I know written by an American poet which can truly be called a real poem and can therefore become the spiritual asset of a nation,
is Edgar Allan Poe's ``\emph{Annabel Lee}.''

Now many people might think that I was only joking when in one of my articles lately I said
that reading the English nursery rhyme about Taffy the Welshman could help to educate the soul of a man.
But in all seriousness I meant what I said.
For poems like the English nursery rhymes are real poems,
but of course, real poems for children: and real poems have a magic in them which,
as Matthew Arnold says of Homer's poetry, can transmute and transform man.
If you don't believe this, you just ask a Japanese geisha to repeat the famous Chinese poem in Japanese,
``Tsuki ochi karasu naite shimo ten ni mitsu,'' and you will see that,
with her eyes suddenly becoming illumined and all her features suffused with ecstasy, she becomes,
for the moment, quite an other woman, more beautiful and charming than she really is. 

In other words, things like the English nursery rhymes are really the expression, in a small way, of a nation's spiritual life and form a part of its civilization.
Now, the fact that the Americans as a nation have no nursery rhymes is to me a proof positive
that the Americans as a nation have no spiritual life;
that they are still a nation, as I have said,
without any spiritual asset of civilization. 

Finally, then, if the United States were destroyed tomorrow,
I want to ask what great spiritual thing have the Americans as a nation done which they can leave behind them to show to men of after generations that they were once a nation with a civilization.
In my opinion, the only spiritual asset of the American Nation,
the only really spiritual things which the Americans as a nation have done that,
if they, as a nation were destroyed tomorrow,
will be remembered by men of after generations
-- are the work of Poe's ``Annabel Lee'' and the music of the plantation songs of the negroes in America.

Perhaps some people will say to me,
``What about the President Wilson's `Fourteen Points,'
which, like the Ten Commandments of Moses, will found a new religion to make the world safe for democracy,
and thus bring in the millennium?''
In reply to this I say that these Fourteen Commandments of President Wilson were made only two years ago,
and now even Mr. Lkoyd George has completely forgotten them! 

Last of all, speaking of President Wilson reminds me
that I have not asked what really great men have the Americans as a nation produced?
Everybody will, of course, answer, George Washington;
but even at the risk of mortally offending all my American friends,
I must say that to me George Washington,
although an admirable man in many ways, had yet, unfortunately,
like the American Pilgrim Fathers, a little too much of the ``moral prig'' in him to be really a great man
-- a really great character like the French Bayard or the English Sir Walter Raleigh and Philip Sydney.
Indeed, as Matthew Arnold, speaking of the American Pilgrim Fathers, says,
``Notwithstanding the mighty results of the Pilgrim Fathers' voyage,
they and their standard of perfection are rightly judged when we figure to ourselves Shakespeare and Virgil accompanying them on their voyage and think what intolerable company Shakespeare and Virgil would have found them''
-- so I , for my part, if I had to take a trip to Merry England in Shakespeare's time or in modern times to Japan, a man like George Washington would be the last man I would take with me as a companion,
for one could not think of taking such a man to the Windsor ``Garter Inn'' and introducing him to Mrs. Ford, Mrs. Anne Page, Bardolf, Pistol, Nym and others of that merry company, nor to a Geisha house in Japan! 

A Scotchman I met at the Pekin Pavilion the other night, who had lately returned from Scotland,
when I asked him whether he thought there was any likelihood of Scotland becoming dry,
like America, answered: ``No, certainly not.'' I asked him why.
``Because,'' he said proudly,
``Scotland is a country with a Robbie Burns. America can become dry because America is country which has never had a Robbie Burns.''
In the same way, to foreigners who, when I tell them that a big country like China cannot be a republic, say to me that America, also a big country, is a republic,
I answer, ``America can be a Republic because the Americans, unlike us Chinese, are a nation without any civilization, just as my Scotch friend says,
America can become dry because she is a country without a Robert Burns.''
In fact, China, I may say here, has become a republic now because the poor demented imbecile Republican Chinaman today has lost his civilization as he has lost his queue.
The reason again why the republic in China cannot work now,
is because not all Chinese have lost their civilization.

Now before I conclude, I want to say that my object in writing this article is not to abuse the American people.
The American people, I have tried to show, are still a nation without any civilization and that is because they are still a young nation.
To be young, as William Pitt in his famous speech in the British Parliament once said, is not an ``atrocious crime.'' Indeed because the Americans are a young nation living in a big country like America,
they have such great potentialities that I am inclined to think that,
if civilization can be saved after this great world upheaval,
the three great future empires of the world will be American, Russia and China. 

My object in writing this article, I say again, is not to abuse the American people.
My object is to tell people that the only way to save civilization
-- the first thing you must do if you want to save civilization
-- is to know what civilization is.

Indeed what made me write this article is because I read lately an essay written by the Prime Minister of Japan,
Mr. Hara, in which he said that he wanted to amalgamate the civilization of the East and West.
It seems to me, when he said that, the Prime Minister of Japan,
if I may be permitted to say so, does not know what is civilization.
For a civilization is either a true civilization or a false or,
as the Japanese say, a magai make-believe civilization:
there is no East or West. 

Confucius in his time was so sick of hearing people talk of civilization that he once said,
``Civilization, Civilization, is the cry now:
but are carrying fine jades and wearing silk dresses the whole of civilization?''
In the same way I take the liberty here of asking men like the Japanese Prime Minister who talk of amalgamating civilizations:
are wearing high collars, cutting off queues, building European houses,
riding in motor cars and erecting statues such as one sees in the streets of Tokio
-- the whole of civilization, or even civilization at all? Indeed, Matthew Arnold, speaking of Christianity
-- curiously using the very same words which the Chinese poet Su Tung-Po of the Sung dynasty used when speaking of Confucianism -- says,
``Christianity is, first and above all, a temper, a disposition
-- so I want to point out here that civilization also is first and above all, a state of the mind and heart:
a spiritual life. In the true sense of civilization, an ordinary Japanese Gelthar is more civilized than most American professors with all their pragmatic philosophy and political science.
In fact, civilization in its essence, is not dress, house, furniture, machine, ship or gun, but
-- gentleness of mind and heart or, in the words of the great Japanese Shogun which I have already quoted -- a gentle and tender heart.
Last of all then, I would like again the remind the Japanese people, who, in my opinion,
are now the real Se-i Shogun or Markgraf Nation, Military Guardian of Civilization in the Far East,
of the words of their great dead Shogun Tokugawa lyeyasu, who, when speaking to his grandson, was really speaking to the Japanese Nation today. 

\section{``Mad Monk of Russia''}
Recent news dispatches from Riga, Latvia, state that Iliodor,
the ``Mad Monk of Russia,'' Rasputin's former friend, is back in Tsaritzyn,
where the monastery built from the offerings of the people to him is located.
His presence in Russia was predicted by himself in a book published here by the Century Company in which
he revealed the story of his relations with Rasputin,
with the Czar and with the spies.
He stated his intention of going back,
believing his mission was to restore religion of the people under the revolution.




\chapter{辜鴻銘的西文學習法}

%\chapterauthors{餘一彥
  %\chapteraffil{(affiliation or affiliations)}
%}
餘一彥

〔編者按:本文摘自兒童中西文化導讀通訊第一期,40-42頁〕 


辜鴻銘,精通九國的語言文化,國學造詣極深,曾獲贈博士學位13個之多。他的思想影響跨越20世紀的東西方,是一位學貫中西、文理兼通的學者,又是近代中學西漸史上的先驅人物。

辜鴻銘I0歲時就隨他的義父一一英人布朗踏上蘇格蘭的土地,被送到當地一所著名的中學,受極嚴格的英國文學訓練。課餘的時間,布朗就親自教辜鴻銘學習德文。布朗的教法略異於西方的傳統,倒像是中國的私墊。他要求辜鴻銘隨他一起背誦歌德的長詩《浮士德》。布朗告訴辜鴻銘:"在西方有神人,卻極少有聖人。神人生而知之,聖人學而知之。西方只有歌德是文聖,毛奇是武聖。要想把德文學好,就必須背熟歌德的名著《浮士德》。"他總是比比劃劃地邊表演邊朗誦,要求辜鴻銘模仿著他的動作背誦,始終說說笑笑,輕鬆有趣。辜鴻銘極想知道《浮士德》書裏講的是什麼,但布朗堅持不肯逐字逐句地講解。他說:"只求你說得熟,並不求你聽得懂。聽懂再背,心就亂了,反倒背不熟了。等你把《浮士德》倒背如流之時我再講給你聽吧!"半年多的工夫辜鴻銘便稀裏糊塗地把一部《浮士德》大致背了下來。

第二年布朗才開始給辜鴻銘講解《浮士德》。他認為越是晚講,瞭解就越深,因為經典名作不同於一般著作,任何人也不能夠一聽就懂。這段時間裏辜鴻銘並沒有停頓對《浮士德》的記誦,已經可謂"倒背如流"了。

學完《浮士德》,辜鴻銘開始學"莎士比亞"的戲劇。布朗為辜鴻銘定下了半月學一部戲劇的計劃。八個月之後,見辜鴻銘記誦領會奇快,計劃又改為半月學三部。這樣大約不到一年,辜鴻銘已經把"莎士比亞"的37部戲劇都記熟了。

布朗認為辜鴻銘的英文和德文水準已經超過了一般大學畢業的文學士,將來足可運用自如了。但辜鴻銘只學了詩和戲劇,尚未正式涉及散文。布朗安排辜鴻銘讀卡萊爾的歷史名著《法國革命》。辜鴻銘此次基本轉入自學,自己慢慢讀慢慢背,遇有不懂的詞句再去請教別人。但只讀了三天,辜鴻銘就哭了起來。布朗吃驚地問:"怎樣了"辜鴻銘回答說:"散文不如戲劇好背。"布朗又問辜鴻銘背誦的進度,發現他每天讀三頁,於是釋然:"你每天讀得太多了。背誦散文作品每天半頁到一頁就夠多了。背誦散文同樣是求熟不求快,快而不熟則等於沒學。"

辜鴻銘所在的中學課業本來是極繁重的,但由於辜鴻銘各科在布朗身邊都提前打下了基礎,整個學習的過程便顯得毫不費力。學校的功課既然順利,沒事的辜鴻銘便接著記誦卡萊爾的《法國革命》。他越讀越有興致,可是讀多了便無法背熟。若按布朗的要求慢慢來,又控制不了自己的好奇心。就這樣時快時慢地把卡萊爾的《法國革命》讀完了。後來辜鴻銘終於徵得義父的同意,可以隨便閱讀義父布朗家中的藏書了。有許多書,辜鴻銘並沒有打算背熟,但也在不經意間"過目成誦"了。

布朗對義子的寄望極高。他曾告訴辜鴻銘:"我若有你的聰明,甘願作一個學者,拯救人類;不作一個百萬富翁,造福自己。讓我告訴你,現在歐洲國家和美國都想侵略中國,但是歐洲各國和美國的學者卻多想學習中國。我希望你能夠學通中西,就是為了教你擔起強化中國,教化歐美的重任,能夠給人類指出一條光明的大道,讓人能過上真正是人的生活!"

依照布朗的計劃,辜鴻銘應該先在英國學文、史、哲學及社會學,然後再到德國學習科學。學成之後才可以回中國修習傳統文化。布朗當初確實沒有看錯,辜鴻銘十四歲時,學術造詣就己經非一般人所能比。他只用了短短四年的時間,不僅初步完成了布朗擬定的家庭教學計劃,而且基本上修完了所在中學的各門主要課程。布朗不禁暗自為義子的聰明而感到驕傲。辜鴻銘在學校裏初步掌握了拉丁文和希臘文,其他課程的成績也都很出色,已經可以申請畢業了。

大約在1872年春季,辜鴻銘正式入愛丁堡大學就讀。辜鴻銘在愛丁堡大學的專修科為英國文學,同時兼修拉丁文、希臘文、數學、形而上學、道德哲學、自然哲學、修辭學等科目。辜鴻銘在學習拉丁文、希臘文時又不知暗自哭了多少次。他立志遍讀愛丁堡大學圖書館所藏希臘、拉丁文的文、史、哲名著。剛開始時,讀多少頁便背誦多少頁,還沒覺出多麼困難;後來隨著閱讀量的逐漸增大,漸漸感到吃不消了。他要自己堅持,再堅持,一定要一路背誦下去。辜鴻銘晚年憶及此事時曾悅:"說也奇怪,一通百通,像一條機器線,一拉開到頭。"到後來,不僅希臘、拉丁文,即如法、俄、意各國的語言、文學,辜鴻銘也能做到一學就會,觸類旁通。據悅辜鴻銘回國後,除本國語言外,尚能操九種文字與人交流,則其基礎主要是在愛丁堡大學讀書時打下的。

《論語.季氏》有雲:"生而知之者,上也。學而知之者,次也。困而學之,又其次也。困而不學,民斯為下矣。"至於"困"字的意思,舊註謂"有所不通",錢穆先生解作"經歷困境",辜鴻銘則自謂"吃不消「,他晚年曾對人說:"其實我讀書時主要的還是堅持『困而學之』的方法。久而久之不難掌握學習藝術,達到'不亦悅乎'的境地。旁人只看見我學習得多,學習得快,他們不知道我是用眼淚換來的!有些人認為記憶好壞是天生的,不錯,人的記憶力確實有優劣之分,但是認為記憶力不能增加是錯誤的。人心愈用而愈靈!"辜鴻銘憶起讀書時的往事,不禁慨嘆道:"困而不學,民斯為下矣!"(兆文鈞《辜鴻銘先生對我講述的往事》)則當時人們多認為辜鴻銘的博學在於他的天賦聰明,辜鴻銘自己是不承認的。

1871年4月,辜鴻銘以優秀的成績通過了所有相關科目的考試,在英國文學方面的學位考試中又表現非凡,順利獲得了愛丁堡大學文學碩士學位。這一年辜鴻銘僅20歲。

辜鴻銘自萊比錫大學畢業後,又赴巴黎短期進修法文。布朗又為辜鴻銘聯繫入巴黎大學,意在讓他學一些法學和政治學。其實當時辜鴻銘只22歲即已遍學科學、文學、哲學,並熟諳各國語言,造詣確非一般中國留學生可比。辜鴻銘以極快的速度讀完了巴黎大學整學期的講義和參考書,除偶爾去學校上點感興趣的課以外,辜鴻銘每天都抽一點時間教他的女房東學希臘文。從剛開始教她學希臘文字母那天起,辜鴻銘就教她背誦幾句《伊利亞特》。他的女房東笑著說:"你的教法真新鮮,沒聽說過。"於是,辜鴻銘就把布朗教自己背誦《浮士德》和莎翁戲劇的經過講給她聽。她說:"好,我這樣學下去。"辜鴻銘稅:"等你背熟一本,你就要背兩本,擋都擋不住。"

辜鴻銘的女房東常常拿著《伊利亞特》來到他的房間,把學過的詩句背給他聽,請求他的指點。辜鴻銘的教法果然有效,他的女房東在希臘文方面進展神速。許多客人見辜鴻銘教她學希臘文的方法與眾不同,都大為驚訝。

辜鴻銘後來曾對晚清直隸布政使淩福彭說:"學英文最好像英國人教孩子一樣的學,他們從小都學會背誦兒歌,稍大一點就教背詩背聖經,像中國人教孩子背四書五經一樣。"從辜鴻銘教他的女房東學希臘文的過程中可以看出,背誦《伊利亞特》的要旨即在於創造了一種真實的誦讀感受,如在希臘國土受希臘純正的啟蒙教育一般。此法乍看強度大,難度亦大,其實則不然。若由字母而單詞,再簡單拼句,則學習者在心理上就產生學外國語言的隔閡情緒了。辜鴻銘還依此法教會了他的女房東簡易的拉丁文,也不過三兩個月的工夫而已。

辜鴻銘深厚的西方素養極得益於童年背誦《浮士德》、《莎士比亞》的經歷。他後來在北京大學教英詩時,有學生向他請教掌握西文的妙法,他答曰:"先背熟一部名家著作作根基。"辜鴻銘曾說:"今人讀英文十年,開目僅能閱報,伸紙僅能修函,皆由幼年讀一貓一狗之式教科書,是以終其身只有小成。"他主張"中國私墊教授法,以開蒙未久,即讀四書五經,尤須背誦如流水也。"




\backmatter%%%%%%%%%%%%%%%%%%%%%%%%%%%%%%%%%%%%%%%%%%%%%%%%%%%%%%%

%\include{glossary}
%\Extrachap{Glossary}
%Use the template \emph{glossary.tex} together with the Springer document class SVMono (monograph-type books) or SVMult (edited books) to style your glossary\index{glossary} in the Springer layout.

%\runinhead{glossary term} Write here the description of the glossary term. Write here the description of the glossary term. Write here the description of the glossary term.

%\Extrachap{Solutions}
%\include{solutions}
%\section*{Problems of Chapter~\ref{intro}}
%\begin{sol}{prob1}
%The solution\index{problems}\index{solutions} is revealed here.
%\end{sol}
%\begin{sol}{prob2}
%\textbf{Problem Heading}\\
%(a) The solution of first part is revealed here.\\
%(b) The solution of second part is revealed here.
%\end{sol}


\printindex

%%%%%%%%%%%%%%%%%%%%%%%%%%%%%%%%%%%%%%%%%%%%%%%%%%%%%%%%%%%%%%%%%%%%%%

\end{document}





