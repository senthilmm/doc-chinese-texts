
\chapter[The Spirit of the Chinese People]
{The Spirit of the Chinese People}

\prologue{
A Paper that was to have been read before the Oriental Society of Peking
}{}

Let me first of all explain to you what I propose, with your permission, this afternoon to discuss.
The subject of our paper I have called ``The Spirit of the Chinese people.''
I do not mean here merely to speak of the character or characteristics of the Chinese people.
Chinese characteristics have often been described before,
but I think you will agree with me that such description or enumeration of the characteristics of the Chinese people hitherto
have given us no picture at all of the inner being of the Chinaman.
Besides, when we speak of the character or characteristics of the Chinese,
it is not possible to generalize.
The character of the Northern Chinese, as you know,
is as different from that of the Southern Chinese as the character of the Germans is different from that of the Italians.

But what I mean by the spirit of the Chinese people, is the spirit by which the Chinese people live, something constitutionally distinctive in the mind,
temper and sentiment of the Chinese people which distinguishes them from all other people, especially from those of modem Europe and America.
Perhaps I can best express what I mean by calling the subject of our discussion the Chinese type of humanity, or, to put it in plainer and shorter words, the real Chinaman.

Now, what is the real Chinaman? That, I am sure, you will all agree with me,
is a very interesting subject, especially at the present moment,
when from what we see going on around us in China today,
it would seem that the Chinese type of humanity -- the real Chinaman -- is going to disappear and,
in his place, we are going to have a new type of humanity -- the progressive or modern Chinaman.
In fact I propose that before the real Chinaman,
the old Chinese type of humanity, disappears altogether from the world
we should take a good last look at him
and see if we can find anything organically distinctive in him
which makes him so different from all other people
and from the new type of humanity which we see rising up in China today.


Now the first thing, I think, which will strike you in the old Chinese type of humanity is that there is nothing wild, savage or ferocious in him. Using a term which is applied to animals, we may say of the real Chinaman that he is a domesticated creature.
Take a man of the lowest class of the population in China and, I think, you will agree with me that there is less of animality in him, less of the wild animal, of what the Germans call \emph{Rohheit}, than you will find in a man of the same class in a European society. In fact, the one word, it seems to me, which will sum up the impression which the Chinese type of humanity makes upon you is the English word ``gentle.''
By gentleness I do not mean softness of nature or weak submissiveness. ``The docility of the Chinese,'' says the late Dr. D. J. Macgowan, ``is not the docility of a broken-hearted, emasculated people. ''
But by the word ``gentle'' I mean absence of hardness, harshness, roughness, or violence, in fact of anything which jars upon you. There is in the true Chinese type of humanity an air, so to speak, of a quiet, sober, chastened mellowness, such as you find in a piece of well-tempered metal. Indeed the very physical and moral imperfections of a real Chinaman are, if not redeemed, at least softened by this quality of gentleness in him.
The real Chinaman may be coarse, but there is no grossness in his coarseness. The real Chinaman may be ugly, but there is no hideousness in his ugliness. The real Chinaman may be vulgar, but there is no aggressiveness, no blatancy in his vulgarity.
The real Chinaman may be stupid, but there is no absurdity in his stupidity.
The real Chinaman may be cunning, but there is no deep malignity in his cunning. In fact what I want to say is, that even in the faults and blemishes of body, mind and character of the real Chinaman,
there is nothing which revolts you. It is seldom that you will find a real Chinaman of the old school, even of the lowest type, who is positively repulsive.

I say that the total impression which the Chinese type of humanity makes upon you is that he is gentle, that he is inexpressibly gentle.
When you analyse this quality of inexpressible gentleness in the real Chinaman, you will find that it is the the product of a combination of two things, namely, sympathy and intelligence.
I have compared the Chinese type of humanity to a domesticated animal. Now what is that which makes a domesticated animal so different from a wild animal?
It is something in the domesticated animal which we recognise as distinctively human.
But what is distinctively human as distinguished from what is animal?
It is intelligence.
But the intelligence of a domesticated animal is not a thinking intelligence.
It is not an intelligence which comes to him from reasoning.
Neither does it come to him from instinct, such as the intelligence of the fox,
-- the vulpine intelligence which knows where eatable chickens are to be found. This intelligence which comes from instinct, of the fox, all, -- even wild, animals have.
But this, what may be called \emph{human} intelligence of a domesticated animal is something quite different from the vulpine or animal intelligence.
This intelligence of a domesticated animal is an intelligence which comes not from reasoning nor from instinct, but from sympathy, from a feeling of love and attachment.
A thorough-bred Arab horse understands his English master not because he has studied English grammar nor because he has an instinct for the English language, but because he loves and is attached to his master.
This is what I call human intelligence, as distinguished from mere vulpine or animal intelligence.
It is the possession of this human quality which distinguishes domesticated from wild animals.
In the same way, I say, it is the possession of this sympathetic and true human intelligence, which gives to the Chinese type of humanity,
to the real Chinaman, his inexpressible gentleness.

I once read somewhere a statement made by a foreigner who had lived in both countries, that the longer a foreigner lives in Japan the more he dislikes the Japanese, whereas the longer a foreigner lives in China the more he likes the Chinese.
I do not know if what is said of the Japanese here, is true.
But, I think, all of you who have lived in China will agree with me that what is here said of the Chinese is true.
It is well-known fact that the liking -- you may call it the taste for the Chinese -- grows upon the foreigner the longer he lives in this country.
There is an indescribable something in the Chinese people which, in spite of their want of habits of cleanliness and refinement, in spite of their many defects of mind and character, makes foreigners like them as foreigners like no other people.
This indescribable something which I have defined as gentleness, softens and mitigates, if it does not redeem, the physical and moral defects of the Chinese in the hearts of foreigners.
This gentleness again is, as I have tried to show you, the product of what I call sympathetic or true human intelligence -- an intelligence which comes not from reasoning nor from instinct, but from sympathy -- from the power of sympathy.
Now what is the secret of the power of sympathy of the Chinese people?

I will here venture to give you an explanation -- a hypothesis,
if you like to call it so -- of the secret of this power of sympathy in the Chinese people and my explanation is this.
The Chinese people have this power, this strong power of sympathy, because they live wholly, or almost wholly, a life of the heart.
The whole life of Chinaman is a life of feeling -- not feeling in the sense of sensation which comes from the bodily organs, nor feeling in the sense of passions which flow, as you would say, from the nervous system, but feeling in the sense of emotion or \emph{human affection} which comes from the deepest part of our nature -- the heart or soul.
Indeed I may say here that the real Chinaman lives so much a life of emotion or human affection, a life of the soul, that he may be said sometimes to neglect more than he ought to do, even the necessary requirements of the life of the senses of a man living in this world composed of body and soul.
That is the true explanation of the insensibility of the Chinese to the physical discomforts of unclean surroundings and want of refinement.
But that is neither here nor there.

The Chinese people, I say, have the power of sympathy because they live wholly a life of the heart -- a life of emotion or human affection. Let me here, first of all, give you two illustrations of what I mean by living a life of the heart. My first illustration is this. Some of you may have personally known an old friend and colleague of mine in Wuchang -- known him when he was Minister of the Foreign Office here in Peking -- Mr. Liang Tun-yen, Mr. Liang told me, when he first received the appointment of the Customs Taotai of Hankow, that what made him wish and strive to become a great mandarin, to wear the red button, and what gave him pleasure then in receiving this appointment, was not because he cared for the red button, not because he would henceforth be rich and independent,  -- and we were all of us very poor then in Wuchang,  -- but because he wanted to rejoice, because this promotion and advancement of his would gladden the heart of his old mother in Canton. That is what I mean when I say that the Chinese people live a life of the heart -- a life of emotion or human affection.

My other illustration is this. A Scotch friend of mine in the Customs told me he once had a Chinese servant who was a perfect scamp, who lied, who ``squeezed, '' and who was always gambling, but when my friend fell ill with typhoid fever in an out-of-the-way port where he had no foreign friend to attend to him, this awful scamp of a Chinese servant nursed him with a care and devotion which he could not have expected from an intimate friend or near relation. Indeed I think what was once said of a woman in the Bible may also be said, not only of the Chinese servant, but of the Chinese people generally: -- ``Much is forgiven them, because they love much. '' The eyes and understanding of the foreigner in China see many defects and blemishes in the habits and in the character of the Chinese, but his heart is attracted to them, because the Chinese have a heart, or, as I said, live a life of the heart -- a life of emotion or human affection.

Now we have got, I think, a clue to the secret of sympathy in the Chinese people -- the power of sympathy which gives to the real Chinaman that sympathetic or true human intelligence, making him so inexpressibly gentle. Let us next put this clue or hypothesis to the test. Let us see whether with this clue that the Chinese people live a life of the heart we can explain not only detached facts such as the two illustrations I have given above, but also general characteristics which we see in the actual life of the Chinese people.

First of all let us take the Chinese language. As the Chinese live a life of the heart, the Chinese language, I say, is also a language of the heart. Now it is a well-known fact that children and uneducated persons among foreigners in China learn Chinese very easily, much more so than grown-up and educated persons. What is the reason of this? The reason, I say, is because children and uneducated persons think and speak with the language of the heart, whereas educated men, especially men with the modern intellectual education of Europe, think and speak with the language of the head or intellect. In fact, the reason why educated foreigners find it so difficult to learn Chinese, is because they are too educated, too intellectually and scientifically educated. As it is said of the Kingdom of Heaven, so it may also be said of the Chinese language: -- ``Unless you become as little children, you cannot learn it. ''

Next let us take another well-known fact in the life of the Chinese people.
The Chinese, it is well-known, have wonderful memories.
What is the secret of this?
The secret is: the Chinese remember things with the heart and not with the head.
The heart with its power of sympathy, acting as glue,
can retain things much better than the head or intellect which is hard and dry.
It is, for instance, also for this reason that we;
all of us, can remember things which we learnt when we were children much better than we can remember things which we learnt in mature life.
As children, like the Chinese, we remember things with the heart and not with the head.

Let us next take another generally admitted fact in the life of the Chinese people -- their politeness.
The Chinese are, it has often been remarked, a peculiarly polite people.
Now what is the essence of true politeness?
It is consideration for the feelings of others.
The Chinese are polite because, living a life of the heart,
they know their own feelings and that makes it easy for them to show consideration for the feelings of others. The politeness of the Chinese,
although not elaborate like the politeness of the Japanese,
is pleasing because it is, as the French beautifully express it,
\emph{la politesse du coeur}, the politeness of the heart.
The politeness of the Japanese, on the other hand,
although elaborate, is not so pleasing,
and I have heard some foreigners express their dislike of it,
because it is what may be called a rehearsal politeness
-- a politeness learnt by heart as in a theatrical piece.
It is not a spontaneous politeness which comes direct from the heart.
In fact the politeness of the Japanese is like a flower without fragrance,
whereas the politeness of a really polite Chinese has a perfume like the aroma of a precious ointment
-- \emph{instar unguenti fragrantis} --  which comes from the heart.

Last of all, let us take another characteristic of the Chinese people, by calling attention to which the Rev. Arthur Smith has made his reputation, viz. : -- want of exactness. Now what is the reason for this want of exactness in the ways of the Chinese people? The reason, I say again, is because the Chinese live a life of the heart. The heart is a very delicate and sensitive balance. It is not like the head or intellect, a hard, stiff, rigid instrument. You cannot with the heart think with the same steadiness, with the same rigid exactness as you can with the head or intellect. At least, it is extremely difficult to do so. In fact, the Chinese pen or pencil which is a soft brush, may be taken as a symbol of the Chinese mind. It is very difficult to write or draw with it, but when you have once mastered the use of it, you will, with it, write and draw with a beauty and grace which you cannot do with a hard steel pen.

Now the above are a few simple facts connected with the life of the Chinese people which anyone, even without any knowledge of Chinese, can observe and understand, and by examining these facts, I think, I have made good my hypothesis that the Chinese people live a life of the heart.

Now it is because the Chinese live a life of the heart,
the life of a child, that they are so primitive in many of their ways.
Indeed, it is a remarkable fact that for a people who have lived so long in the world as a great nation,
the Chinese people should to this day be so primitive in many of their ways.
It is this fact which has made superficial foreign students of China think
that the Chinese have made no progress in their civilisation
and that the Chinese civilisation is a stagnant one. Nevertheless,
it must be admitted that,
as far as pure intellectual life goes,
the Chinese are, to a certain extent, a people of arrested development.
The Chinese, as you all know, have made little or no progress not only in the physical, but also in the pure abstract sciences such as mathematics,
logic and metaphysics.
Indeed the very words ``science'' and ``logic'' in the European languages have no exact equivalent in the Chinese language.
The Chinese, like children who live a life of the heart,
have no taste for the abstract sciences,
because in these the heart and feelings are not engaged.
In fact, for everything which does not engage the heart and feelings,
such as tables of statistics, the Chinese have a dislike amounting to aversion.
But if tables of statistics and the pure abstract sciences fill the Chinese with aversion,
the physical sciences as they are now pursued in Europe,
which require you to cut up and mutilate the body of a living animal in order to verify a scientific theory,
would inspire the Chinese with repugnance and horror.

The Chinese, I say, as far as pure intellectual life goes,
are to a certain extent, a people of arrested development.
The Chinese to this day live the life of a child, a life of the heart. In this respect, the Chinese people, old as they are as a nation, are to the present day, a nation of children. But then it is important you should remember that this nation of children, who live a life of the heart, who are so primitive in many of their ways, have yet a \emph{power of mind and rationality} which you do not find in a primitive people, a power of mind and rationality which has enabled them to deal with the complex and difficult problems of social life, government and civilisation with a success which, I will venture to say here, the ancient and modern nations of Europe have not been able to attain
-- a success so signal that they have been able practically and
actually to keep in peace and order a greater portion of the population of the Continent of Asia under a great Empire.

In fact, what I want to say here, is that the wonderful peculiarity of the Chinese people is not that they live a life of the heart.
All primitive people also live a life of the heart. The Christian people of medieval Europe, as we know, also lived a life of the heart. Matthew Arnold says: -- ``The poetry of medieval Christainity lived by the heart and imagination.''
But the wonderful peculiarity of the Chinese people, I want to say here, is that, while living a life of the heart, the life of a child, they yet have a power of mind and rationality
which you do not find in the Christian people of medieval Europe or in any other primitive people. In other words, the wonderful peculiarity of the Chinese is that for a people, who have lived so long as a grown-up nation, as a nation of adult reason, they are yet able to this day to live the life of a child
-- a life of the heart.
Instead, therefore, of saying that the Chinese are a people of arrested development, one ought rather to say that the Chinese are a people who never grow old.In short the wonderful peculiarity of the Chinese people as a race, is that they possess the secret of perpetual youth.

Now we can answer the question which we asked in the beginning: -- What is the real Chinaman? The real Chinaman, we see now, is a man who lives the life of a man of adult reason with the heart of a child. In short the real Chinaman is a person \emph{with the head of a grown-up man and the heart of a child}. The Chinese spirit, therefore, is a spirit of perpetual youth, the spirit of national immortality. Now what is the secret of this national immortality in the Chinese people? You will remember that in the beginning of this discussion I said that what gives to the Chinese type of humanity -- to the real Chinaman -- his inexpressible gentleness is the possession of what I called sympathetic or true human intelligence. This true human intelligence, I said, is the product of a combination of two things, sympathy and intelligence. It is a working together in harmony of the heart and head. In short it is a happy union of soul with intellect. Now if the spirit of the Chinese people is a spirit of perpetual youth, the spirit of national immortality, the secret of this immortality is this happy union of soul with intellect.

You will now ask me where and how did the Chinese people get this secret of national immortality -- this happy union of soul with intellect, which has enabled them as a race and nation to live a life of perpetual youth? The answer, of course, is that they got it from their civilisation. Now you will not expect me to give you a lecture on Chinese civilisation within the time at my disposal. But I will try to tell you something of the Chinese civilisation which has a bearing on our present subject of discussion.

Let me first of all tell you that there is, it seems to me,
one great fundamental difference between the Chinese civilisation and the civilisation of modern Europ.
Here let me quote an admirable saying of a famous living art critic, Mr. Bernard Berenson.
Comparing European with Oriental art, Mr. Berenson says:
-- ``Our European art has the fatal tendency to become science and
we hardly possess a masterpiece which does not bear the marks of
having heen a \emph{battlefield for divided interests}. ''
Now what I want to say of the European civilisation is that it is,
as Mr. Berenson says of European art, a battlefield for divided interests;
a continuous warfare for the divided interests of science and art on the one hand,
and of religion and philosophy on the other;
in fact a terrible battlefield where the head and the heart
-- the soul and the intellect -- come into constant conflict.
In the Chinese civilisation, at least for the last years,
there is no such conflict.
That, I say, is the one great fundamental difference between the Chinese civilisation and that of modern Europe.

In other words, what I want to say, is that in modern Europe, the people have a religion which satisfies their heart, but not their head, and a philosophy which satisfies their head but not their heart. Now let us look at China. Some people say that the Chinese have no religion. It is certainly true that in China even the mass of the people do not take seriously to religion. I mean religion in the European sense of the word. The temples, rites and ceremonies of Taoism and Buddhism in China are more objects of recreation than of edification; they touch the aesthetic sense, so to speak, of the Chinese people rather than their moral or religious sense; in fact, they appeal more to their imagination than to their heart or soul. But instead of saying that the Chinese have no religion, it is perhaps more correct to say that the Chinese do not want -- do not feel the need of religion.

Now what is the explanation of this extraordinary fact that the Chinese people, even the mass of the population in China, do not feel the need of religion? It is thus given by an Englishman. Sir Robert K. Douglas, Professor of Chinese in the London University, in his study of Confucianism, says: -- ``Upwards of forty generations of Chinamen have been absolutely subjected to the dicta of one man. Being a Chinaman of Chinamen the teachings of Confucius were specially suited to the nature of those he taught. \emph{The Mongolian mind being eminently phlegmatic and unspeculative}, naturally rebels against the idea of investigating matters beyond its experiences. With the idea of a future life still unawakened, a plain, matter-of-fact system of morality, such as that enunciated by Confucius, was sufficient for all the wants of the Chinese. ''

That learned English professor is right, when he says that the Chinese people do not feel the need of religion, because they have the teachings of Confucius, but he is altogether wrong, when he asserts that the Chinese people do not feel the need of religion because the Mongolian mind is phlegmatic and unspeculative. In the first place religion is not a matter of speculation. Religion is a matter of feeling, of emotion; it is something which has to do with the human soul. The wild, savage man of Africa even, as soon as he emerges from a mere animal life and what is called the soul in him, is awakened,  --  feels the need of religion. Therefore although the Mongolian mind may be phlegmatic and unspeculative, the Mongolian Chinaman, who, I think it must be admitted, is a higher type of man than the wild man of Africa, also has a soul, and, having a soul, must feel the need of religion unless he has something which can take for him the place of religion.

The truth of the matter is,  -- the reason why the Chinese people do not feel the need of religion is because they have in Confucianism a system of philosophy and ethics, a synthesis of human society and civilisation which can take the place of religion. People say that Confucianism is not a religion. It is perfectly true that Confucianism is not a religion in the ordinary European sense of the word. But then I say the greatness of Confucianism lies even in this, that it is not a religion. In fact, the greatness of Confucianism is that, without being a religion, it can take the place of religion; it can make men do without religion.

Now in order to understand how Confucianism can take the place of religion we must try and find out the reason why mankind, why men feel the need of religion. Mankind, it seems to me, feel the need of religion for the same reason that they feel the need of science, of art and of philosophy. The reason is because man is a being who has a soul. Now let us take science, I mean physical science. What is the reason which makes men take up the study of science? Most people now think that men do so, because they want to have railways and aeroplanes. But the motive which impels the true men of science to pursue its study is not because they want to have railways and aeroplanes. Men like the present progressive Chinamen, who take up the study of science, because they want railways and aeroplanes, will never get science. The true men of science in Europe in the past who have worked for the advancement of science and brought about the possibility of building railways and aeroplanes, did not think at all of railways and aeroplanes. What impelled those true men of science in Europe and what made them succeed in their work for the advancement of science, was because they \emph{felt in their souls} the need of understanding the awful mystery of the wonderful universe in which we live. Thus mankind, I say, feel the need of religion for the same reason that they feel the need of science, art and philosophy; and the reason is because man is a being who has a soul, and because the soul in him, which looks into the past and future as well as the present --  not like animals which live only in the present -- feels the need of understanding the mystery of this universe in which they live. Until men understand something of the nature, law, purpose and aim of the things which they see in the universe, they are like children in a dark room who feel the danger, insecurity and uncertainty of everything. In fact, as an English poet says, the burden of the mystery of the universe weighs upon them. Therefore mankind want science, art and philosophy for the same reason that they want religion, to lighten for them ``the burden of the mystery, ....
\begin{quote}
The heavy and the weary weight of All this unintelligible world.''
\end{quote}

Art and poetry enable the artist and poet to see beauty and order in the universe and that lightens for them the burden of this mystery. Therefore poets like Goethe, who says: "He who has art, has religion, " do not feel the need of religion. Philosophy also enables the philosophers to see method and order in the universe, and that lightens for them the burden of this mystery. Therefore philosophers, like Spinoza, "for whom, " it has been said, "the crown of the intellectual life is a transport, as for the saint the crown of the religious life is a transport," do not feel the need of religion. Lastly, science also enables the scientific men to see law and order in the universe, and that lightens for them the burden of this mystery. Therefore scientific men like Darwin and Professor Haeckel do not feel the need of religion.

But for the mass of mankind who are not poets, artists, philosophers or men of science; for the mass of mankind whose lives are full of hardships and who are exposed every moment to the shock of accident from the threatening forces of Nature and the cruel merciless passions of their fellow-men, what is it that can lighten for them the
``burden of the mystery of all this unintelligible world?''
It is religion. But how does religion lighten for the mass of mankind the burden of this mystery? Religion, I say, lightens this burden by giving the mass of mankind a sense of \emph{security} and a sense of \emph{permanence}. In presence of the threatening forces of Nature and the cruel merciless passions of their fellowmen and the mystery and terror which these inspire, religion gives to the mass of mankind a refuge -- a refuge in which they can find a sense of \emph{security}; and that refuge is a belief in some supernatural Being or beings who have absolute power and control over those forces which threaten them. Again, in presence of the constant change, vicissitude and transition of things in their own lives -- birth, childhood, youth, old age and death, and the mystery and uncertainty which these inspire, religion gives to the mass of mankind also a refuge -- a refuge in which they can find a sense of \emph{permanence}; and that refuge is the belief in a future life. In this way, I say, religion lightens for the mass of mankind who are not poets, artists, philosophers or scientific men, the burden of the mystery of all this unintelligible world, by giving them a sense of security and a sense of permanence in their existence. Christ said: " Peace I give unto you, peace which the world cannot give and which the world cannot take away from you." That is what I mean when I say that religion gives to the mass of mankind a sense of security and a sense of permanence. Therefore, unless you can find something which can give to the mass of mankind the same peace, the same sense of security and of permanence which religion affords them, the mass of mankind will always feel the need of religion.

But I said Confucianism, without being a religion can take the place of religion. Therefore, there must be something in Confucianism which can give to the mass of mankind the same sense of security and permanence which religion affords them. Let us now find out what this something is in Confucianism which can give the same sense of security and sense of permanence that religion gives.

I have often been asked to say what Confucius has done for the Chinese nation. Now I can tell you of many things which I think Confucius has accomplished for the Chinese people. But, as to-day I have not the time, I will only here try to tell you of one principal and most important thing which Confucius has done for the Chinese nation -- the one thing he did in his life by which, Confucius himself said, men in after ages would know him, would know what he had done for them. When I have explained to you this one principal thing, you will then understand what that something is in Confucian-ism which can give to the mass of mankind the same sense of security and sense of permanence which religion affords them. In order to explain this, I must ask you to allow me to go a little more into detail about Confucius and what he did.

Confucius, as some of you may know, lived in what is called a period of expansion in the history of China -- a period in which the feudal age had come to an end; in which the feudal, the semi-patriarchal social order and form of government had to be expanded and reconstructed. This great change necessarily brought with it not only confusion in the affairs of the world, but also confusion in men' s minds. I have said that in the Chinese civilisation of the last 2,500 years there is no conflict between the heart and the head. But I must now tell you that in the period of expansion in which Confucius lived there was also in China, as now in Europe, a fearful conflict between the heart and the head. The Chinese people in Confucius' s time found themselves with an immense system of institutions, established facts, accredited dogmas, customs, laws -- in fact, an immense system of society and civilisation which had come down to them from their venerated ancestors. In this system their life had to be carried forward; yet they began to feel -- they had a sense that this system was not of their creation, that it by no means corresponded with the wants of their actual life; that, for them, it was customary, not rational. Now the awakening of this sense in the Chinese people 2,500 years ago was the awakening of what in Europe to-day is called the modern spirit -- the spirit of liberalism, the spirit of enquiry, to find out the why and the wherefore of things. This modern spirit in China then, seeing the want of correspondence of the old order of society and civilisation with the wants of their actual life, set itself not only to reconstruct a new order of society and civilisation, but also to find a basis for this new order of society and civilisation. But all the attempts to find a new basis for society and civilisation in China then failed. Some, while they satisfied the head -- the intellect of the Chinese people, did not satisfy their heart; others, while they satisfied their heart, did not satisfy their head. Hence arose, as I said, this conflict between the heart and the head in China 2,500 years ago, as we see it now in Europe. This conflict of the heart and head in the new order of society and civilisation which men tried to reconstruct made the Chinese people feel dissatisfied with all civilisation, and in the agony and despair which this dissatisfaction produced, the Chinese people wanted to pull down and destroy all civilisation. Men, like Laotzu, then in China as men like Tolstoy in Europe to-day, seeing the misery and suffering resulting from the conflict between the heart and the head, thought they saw something radically wrong in the very nature and constitution of society and civilisation. Laotzu and Chuang-tzu, the most brilliant of Laotzu' s disciples, told the Chinese people to throw away all civilisation. Laotzu said to the people of China: ``Leave all that you have and follow me; follow me to the mountains, to the hermit's cell in the mountains, there to live a true life -- a life of the heart, a life of immortality.''

But Confucius, who also saw the suffering and misery of the then state of society and civilisation, thought he recognised the evil was not in the nature and constitution of society and civilisation, but in the wrong track which society and civilisation had taken, in the wrong basis which men had taken for the foundation of society and civilisation. Confucius told the Chinese people not to throw away their civilisation. Confucius told them that in a true society and true civilisation -- in a society and civilisation with a \emph{true} basis men also could live a true life, a life of the heart. In fact, Confucius tried hard all his life to put society and civilisation on the right track; to give it a true basis, and thus prevent the destruction of civilisation. But in the last days of his life, when Confucius saw that he could not prevent the destruction of the Chinese civilisation -- what did he do? Well, as an architect who sees his house on fire, burning and falling over his head, and is convinced that he cannot possibly save the building, knows that the only thing for him to do is- to save the drawings and plans of the building so that it may afterwards be built again; so Confucius, seeing the inevitable destruction of the building of the Chinese civilisation which he conid not prevent, thought he would save the drawings and plans, and he accordingly saved the drawings and plans of the Chinese civilisation, which are now preserved in the Old Testament of the Chinese Bible -- the five Canonical Books known as the \emph{Wu Ching}, five Canons. That, I say, was a great service which Confucius has done for the Chinese nation -- he saved the drawings and plans of their civilisation for them.

Confucius, I say, when he saved the drawings and plans of the Chinese civilisation, did a great service for the Chinese nation. But that is not the principal, the greatest service which Confucius has done for the Chinese nation. The greatest service he did was that, in saving the drawings and plans of their civilisation, he made a new synthesis, a new interpretation of the plans of that civilisation, and in that new synthesis he gave the Chinese people the true idea of a State -- a true, rational, permanent, absolute basis of a State.

But then Plato and Aristotle in ancient times, and Rousseau and
Herbert Spencer in modern times also made a synthesis of civilisation, and tried to give a true idea of a State. Now what is the difference between the philosophy, the synthesis of civilisation made by the great men of Europe I have mentioned, and the synthesis of civilisation -- the system of philosophy and morality now known as Confu-cianism? The difference, it seems to me, is this. The philosophy of Plato and Aristotle and of Herbert Spencer has not become a religion or the equivalent of a religion, the accepted faith of the masses of a people or nation, whereas Confucianism has become a religion or the equivalent of a religion to even the mass of the population in China. When I say religion here, I mean religion, not in the narrow European sense of the word, but in the broad universal sense. Goethe says: -- ``\emph{Nur saemtliche Menschen erkennen die Natur; nur saemtliche Menschen leben das Menschliche} \footnote{唯有民眾懂得什麼是真正的生活,唯有民眾過著真正的人的生活。}.
Only the mass of mankind know what is real life; only the mass of mankind live a true human life.'' Now when we speak of religion in its broad universal sense, we mean generally a system of teachings with rules of conduct which, as Goethe says, is accepted as true and binding by the mass of mankind, or at least, by the mass of the population in a people or nation. In this broad and universal sense of the word Christianity and Buddhism are religions. In this broad and universal sense, Confucianism, as you know, has become a religion, as its teachings have been acknowledged to be true and its rules of couduct to be binding by the whole Chinese race and nation, whereas the philosophy of Plato, of Aristotle and of Herbert Spencer has not become a religion even in this broad universal sense. That, I say, is the difference between Confucianism and the philosophy of Plato and Aristotle and of Herbert Spencel -- the one has remained a philosophy for the learned, whereas the other has become a religion or the equivalent of a religion for the mass of the whole Chinese nation as well as for the learned of China.

In this broad universal sense of the word, I say Confucianism is a religion just as Christianity or Buddhism is a religion. But you will remember I said that Confucianism is not a religion in the European sense of the word. What is then the difference between Confucianism and a religion in the European sense of the word? There is, of course, the difference that the one has a supernatural origin and element in it, whereas the other has not. But besides this difference of supernatural and non-supernatural, there is also another difference between Confucianism and a religion in the European sense of the word such as Christianity and Buddhism, and it is this. A religion in the European sense of the word teaches a man to be a good \emph{man}. But Confucianism does more than this; Confucianism teaches a man to be a good \emph{citizen}. The Christian Catechism asks: -- "What is the chief end of \emph{man}?" But the Confucian Catechism asks: -- "What is the chief end of a \emph{citizen}?" of man, not in his individual life, but man in his relation with his fellowmen and in his relation to the State? The Christian answers the words of his Catechism by saying:" The chief end of man is to glorify God. " The Confucianist answers the words of his Catechism by saying: ``The chief end of man is to live as a dutiful son and a good citizen.''
Tz\"u Yu, a disciple of Confucius, is quoted in the Sayings and Discourses of Confucius, saying: ``A wise man devotes his attention to the foundation of life -- the chief end of man. When the foundation is laid, wisdom, religion will come. Now to live as a dutiful son and good citizen, is not that the foundation -- the chief end of man as a moral being?''
In short, a religion in the European sense of the word makes it its object to transform man into a perfect ideal man by himself, into a saint, a Buddha, an angel, whereas Confucianism limits itself to make man into a good citizen --  to live as a dutiful son and a good citizen. In other words, a religion
in the European sense of the word says: -- ``If you want to have religion, you must be a saint, a Buddha, an angel;'' whereas Confucian-ism says: -- ``If you live as a dutiful son and a good citizen, you \emph{have} religion.''

In fact, the real difference between Confucianism and religion in the European sense of the word, such as Christianity or Buddhism, is that the one is a personal religion, or what may be called a Church religion, whereas the other is a social religion, or what may be called a State religion. The greatest service, I say, which Confucius has done for the Chinese nation, is that he gave them a true idea of a State. Now in giving this true idea of a State, Confucius made that idea a religion. In Europe politics is a science, but in China, since, Confucius' time, politics is a religion. In short, the greatest service which Confucius has done for the Chinese nation, I say, is that he gave them a Social or State religion. Confucius taught this State religion in a book which he wrote in the very last days of his life, a book to which he gave the name of \emph{Ch'un Ch'iu}(春秋), Spring and Autumn. Confucius gave the name of Spring and Autumn to this book because the object of the book is to give the real moral causes which govern the rise and fall -- the Spring and Autumn of nations. This book might also be called the Latter Day Annals, like the Latter Day Pamphlets of Carlyle. In this book Confucius gave a r\'esum\'e of the history of a false and decadent state of society and civilisation in which he traced all the suffering and misery of that false and decadent state of society and civilisation to its real cause -- to the fact that men had not a true idea of a State; no right conception of the true nature of the duty which they owe to the State, to the head of the State, their ruler and Sovereign. In a way Confucius in this book taught the divine right of kings. Now I know all of you, or at least most of you, do not now believe in the divine right of kings. I will not argue the point with you here. I will only ask you to suspend your judgment until you have heard what I have further to say. In the meantime I will just ask your permission to quote to you here a saying of Carlyle. Carlyle says: ``The right of a king to govern us is either a divine right or a diabolic wrong.'' Now I want you, on this subject of the divine right of kings, to remember and ponder over this saying of Carlyle.

In this book Confucius taught that, as in all the ordinary relations and dealings between men in human society, there is, besides the base motives of interest and of fear, a higher and nobler motive to influence them in their conduct, a higher and nobler motive which rises above all considerations of interest and fear, the motive called \emph{Duty}; so in this important relation of all in human society, the relation between the people of a State or nation and the Head of that State or nation, there is also this higher and nobler motive of Duty which should influence and inspire them in their conduct. Bnt what is the rational basis of this duty which the people in a State or nation owe to the head of the State or nation? Now in the feudal age before Confucius' time, with its semi-patriarchal order of Society and form of Government, when the State was more or less a family, the poeple did not feel so much the need of having a clear and firm basis for the duty which they owe to the Head of the State, because, as they were all members of one clan or family, the tie of kinship or natural affection already, in a way, bound them to the Head of the State, who was also the senior member of their clan or family. But in Confucius' time the feudal age, as I said, had come to an end; when the State had outgrown the family, when the citizens of a State were no longer composed of the members of a clan or family. It was, therefore, then necessary to find a new, clear, rational and firm basis for the duty which the people in a State or nation owe to the Head of the State --  their ruler and sovereign. Now what new basis did Confucius find for this duty? Confucius found the new basis for this duty in the word \emph{Honour}.

When I was in Japan last year the ex-Minister of Education, Baron Kikuchi, asked me to translate four Chinese characters taken from the book in which, as I said, Confucius taught this State religion of his. The four characters were \emph{Ming fen ta yi} (名分大義).
I translated them as the Great Principle of Honour and Duty. It is for this reason that the Chinese make a special distinction between Con-fucianism and all other religions by calling the system of teaching taught by Confucius not a \emph{chiao} (教) -- the general term in Chinese for religion with which they designate other religions, such as Buddhism, Mohammedanism and Christianity -- but the \emph{ming chiao} (名教) -- the religion of Honour. Again the term \emph{chum tzu chih tao} (君子之道) in the teachings of Confucius, translated by Dr. Legge as ``the way of the superior man,'' for which the nearest equivalent in the European languages is moral law -- means literally, the way -- \emph{the Law of the Gentleman}. In fact, the whole system of philosophy and morality taught by Confucius may be summed up in one word: the Law of the Gentleman. Now Confucius codified this law of the gentleman and made it a Religion,  -- a State religion. The first Article of Faith in this State Religion is \emph{Ming fen ta yi} -- the Principle of Honour and Duty -- which may thus be called: A Code of Honour.

In this State religion Confucius taught that the only true, rational, permanent and absolute basis, not only of a State, but of all Society and civilisation, is this law of the gentleman, the sense of honour in man. Now you, all of you, even those who believe that there is no morality in politics -- all of you, I think, know and will admit the importance of this sense of honour in men in human society. But I am not quite sure that all of you are aware of the \emph{absolute} necessity of this sense of honour in men for the carrying on of every form of human society; in fact, as the proverb which says: ``There must be honour even among thieves,'' show -- even for the carrying on of a society of thieves. Without the sense of honour in men, all society and civilisation would on the instant break down and become impossible. Will you allow me to show you how this is so? Let us take, for example, such a trivial matter as gambling in social life. Now unless men when they sit down to gamble all recognise and feel themselves bound by the sense of honour to pay when a certain colour of cards or dice turns up, gambling would on the instant become impossible. The merchants again -- unless merchants recognise and feel themselves bound by the sense of honour to fulfil their contracts, all trading would become impossible. But you will say that the merchant who repudiates his contract can be taken to the law-court. True, but if there were no law-courts, what then? Besides, the law-court -- how can the law-court make the defaulting merchant fulfil his contract? By force. In fact, without the sense of honour in men, society can only be held together for a time by force. But then I think I can show you that force alone cannot hold society permanently together. The policeman who compels the merchant to fulfil his contract, uses force. But the lawyer, magistrate or president of a republic -- how does he make the policeman do his duty? You know he cannot do it by force; but then by what? Either by the sense of honour in the policemen or by \emph{fraud}.

In modem times all over the world to-day -- and I am sorry to say now also in China -- the lawyer, politician, magistrate and president of a republic make the policeman do his duty by fraud. In modem times the lawyer, politician, magistrate and president of a republic tell the policeman that he must do his duty, because it is for the good of society and for the good of his country; and that the good of society means that he, the policeman, can get his pay regularly, without which he and his family would die of starvation. The lawyer, politician or president of a republic who tells the policeman this, I say, uses \emph{fraud}. I say it is fraud, because the good of the country, which for the policeman means fifteen shillings a week, which barely keeps him and his family from starvation, means for the lawyer, politician, magistrate and president of a republic ten to twenty thousand pounds a year, with a fine house, electric light, motor cars and all the comforts and luxuries which the life blood labour of ten thousands of men has to supply him. I say it is fraud because without the recognition of a sense of honour -- the sense of honour which makes the gambler pay the last penny in his pocket to the player who wins from him, \emph{without this sense of honour}, all transfer and possession of property which makes the inequality of the rich and poor in society, as well as the transfer of money on a gambling table, has no justification whatever and no binding force. Thus the lawyer, politician, magistrate or president of a republic, although they talk of the good of society and the good of the country, really depend upon the policeman' s unconscious sense of honour which not only makes him do his duty, but also makes him respect the right of property and be satisfied with fifteen shillings a week, while the lawyer, politician and president of a republic receive an income of twenty thousand pounds a year. I, therefore, say it is fraud because while they thus demand the sense of honour from the policeman; they, the lawyer, politician, magistrate and president of a republic in modem society believe, openly say and act on the principle that there is no morality, no sense of honour in politics.

You will remember what Carlyle, I told you, said -- that the right of a king to govern us is either a divine right or a diabolic wrong. Now this fraud of the modern lawyer, politician, magistrate and president of a republic is what Carlyle calls a diabolic wrong. It is this fraud, this Jesuitism of the public men in modem society, who say and act on the principle that there is no morality, no sense of honour in politics and yet plausibly talk of the good of society and the good of the country; it is this Jesuitism which, as Carlyle says, gives rise to "the widespread suffering, mutiny, delirium, the hot rage of sansculottic insurrections, the cold rage of resuscitated tyrannies, brutal degradation of the millions, the pampered frivolity of the units" which we see in modern society to-day. In short, it is this combination of fraud and force, Jesuitism and Militarism, lawyer and policeman, which has produced Anarchists and Anarchism in modem society, this combination of force and fraud outraging the moral sense in man and producing madness which makes the Anarchist throw bomb and dynamite against the lawyer, politician, magistrate and president of a republic.

In fact, a society without the sense of honour in men, and without morality in its politics, cannot, I say, be held together, or at any rate, cannot last. For in such a society the policeman, upon whom the lawyer, politician, magistrate and president of a republic depend to carry out their fraud, will thus argue with himself. He is told that he must do his duty for the good of society. But he, the poor policeman, is also a part of that society -- to himself and his family, at least, the most important part of that society. Now if by some other way than by being a policeman, perhaps by being an anti-policeman, he can get better pay to improve the condition of himself and his family, that also means the good of society. In that way the policeman must sooner or later come to the conclusion that, as there is .no such thing as a sense of honour and morality in politics, there is then no earthly reason why, if he can get better pay, which means also the good of society -- no reason why, instead of being a policeman, he should not become a revolutionist or anarchist- In a society when the policeman once comes to the conclusion that there is no reason why, if he can get better pay, he should not become a revolutionist or anarchist -- that society is doomed. Mencius said: -- "When Confucius completed his Spring and Autumn Annals" -- the book in which he taught the State religion of his  -- and in which he showed that the society of his time -- in which there was then, as in the world to-day, no sense of honour in public men and no morality in politics -- was doomed; when Confucius wrote that book, ``the Jesuits and anarchists (lit. bandits) of his time, became afraid.'' (乱臣贼子惧)\footnote{Mencius Bk. III, Part II IX, II.}

But to return from the digression, I say, a society without the sense of honour cannot be held together, cannot last. For if, as we have seen, even in the relation between men connected with matters of little or no vital importance such as gambling and trading in human society, the recognition of the sense of honour is so important and necessary, how much more so it must be in the relations between men in human society, which establish the two most essential institutions in that society, the Family and the State. Now, as you all know, the rise of civil society in the history of all nations begins always with the institution of marriage. The Church religion in Europe makes marriage a \emph{sacrament}, i.e.,something sacred and inviolable. The sanction for the sacrament of marriage in Europe is given by the Church and the authority for the sanction is God. But that is only an outward, formal, or so to speak, legal sanction. The true, inner, the really binding sanction for the inviolability of marriage -- as we see it in countries where there is no church religion, is the sense of honour, the law of the gentleman in the man and woman. Confucius says, ``The recognition of the law of the gentleman begins with the recognition of the relation between husband and wife.''\footnote{中庸 -- The Universal order XII 4.}
In other words, the recognition of the sense of honour -- the law of the gentleman -- in all countries where there is civil society, establishes the institution of marriage. The institution of marriage establishes the \emph{Family}.

I said that the State religion which Confucius taught is a Code of Honour, and I told you that Confucius made this Code out of the law of the gentleman. But now I must tell you that long before Confucius' time there existed already in China an undefined and unwritten code of the law of the gentleman. This undefined and unwritten code of the law of the gentleman in China before Confucius' time was known as \emph{li} (礼) the law of propriety, good taste or good manners. Later on in history before Confucius' time a great statesman arose in China -- the man known as the great Law-giver of China, generally spoken of as the Duke of Chou (周公) (B.C. 1135) -- who first defined, fixed, and made a written code of the law of the gentleman, known then in China as \emph{li}, the law of propriety, good taste or good manners. This first written code of the gentleman in China, made by the Duke of Chou, became known as \emph{Chou li} -- the laws of good manners of the Duke of Chou. This Code of the laws of good manners of the Duke of Chou may be consideral as the pre-Confucian religion in China, or, as the Mosaic law of the Jewish nation before Christianity is called, the Religion of the Old Dispensation of the Chinese people. It was this religion of the old dispensation -- the first written code of the law of the gentleman called the Laws of good manners of the Duke of Chou -- which first gave the sanction for the sacrament and inviolability of marriage in China. The Chinese to this day therefore speak of the sacrament of marriage as \emph{Chou Kung Chih Li} (周公之礼) -- the law of good manners of the Duke of Chou. By the institution of the sacrament of marriage, the pre-Confucian or Religion of the Old Dispensation in China established the Family. It secured once for all the stability and permanence of the family in China. This pre-Confucian or Religion of the Old Dispensation known as the laws of good manners of the Duke of Chou in China might thus be called a \emph{Family} religion as distinguished from the State religion which Confucius afterwards taught.

Now Confucius in the State religion which he taught, gave a new Dispensation, so to speak, to what I have called the Family religion which existed before his time. In other words, Confucius gave a new, wider and more comprehensive application to the law of the gentleman in the State religion which he taught; and as the Family religion, or Religion of the Old Dispensation in China before his time instituted the sacrament of marriage, Confucius, in giving this new, wider, and more comprehensive application to the law of the gentleman in the State religion which he taught, instituted a new sacrament. This new sacrament which Confucius instituted, instead of calling it \emph{li} -- the Law of good manners, he called it \emph{ming fen ta yi}, which I have translated as the Great Principle of Honour and Duty or Code of Honour. By the institution of this \emph{ming fen ta yi} or Code of Honour Confucius gave the Chinese people, instead of a Family religion, which they had before -- a State religion.

Confucius, in the State religion which he now gave, taught that, as under the old dispensation of what I have called the Family religion before his time, the wife and husband in a family are bound by the sacrament of marriage, called \emph{Chou Kung Chih Li}, the Law of good manners of the Duke of Chou -- to hold their contract of marriage inviolable and to absolutely abide by it, so under the new dispensation of the State religion which he now gave, the people and their sovereign in every Slate, the Chinese people and their Emperor in China, are bound by this new sacrament called \emph{ming fen ta yi} --  the Great Principle of Honour and Duty or Code of Honour established by this State religion -- to hold the contract of allegiance between them as something sacred and inviolable and absolutely to abide by it. In short, this new sacrament called ming fen ta yi, or Code of Honour which Confucius instituted, is a Sacrament of the Contract of Allegiance, as the old sacrament called \emph{Chou Kung Chih Li}, the Law of Good Manners of the Duke of Chou which was instituted before his time, is a sacrament of marriage. In this way Confucius, as I said, gave a new, wider, and more comprehensive application to the law of the gentleman, and thus gave a new dispensation to what I have called the Family religion in China before his time, and made it a State religion.

In other words, this State religion of Confucius makes a sacrament of the contract of allegiance as the Family Religion in China before his time,
makes a sacrament of the contract of marriage.
As by the sacrament of marriage established by the Family Religion the wife is bound to be absolutely loyal to her husband,
so by this sacrament of the contract of allegiance called \emph{ming fen ta yi},
or Code of Honour established by the State religion taught by Confucius in China,
the people of China are bound to be absolutely loyal to the Emperor.
This sacrament of the contract of allegiance in the State religion taught by Confucius in China might thus be called the \emph{Sacrament or Religion of Loyalty}.
You will remember what I said to you that Confucius in a way taught
the \emph{Divine right of kings}.
But instead of saying that Confucius taught the Divine right of kings I should properly have said that Confucius taught the Divine duty of Loyalty.
This Divine or absolute duty of loyalty to the Emperor in China which Confucius taught derives its sanction,
not as the theory of the Divine right of kings in Europe derives its sanction from the authority of a supernatural Being called God or from some abstruse philosophy,
but from the law of the gentleman
-- the sense of honour in man, the same sense of honour which in all countries makes the wife loyal to her husband.
In fact, the absolute duty of loyalty of the Chinese people to the Emperor which Confucius taught,
derives its sanction from the same simple sense of honour which makes the merchant keep his word and fulfil his contract,
and the gambler play the game and pay his gambling debt.

Now, as what I have called the Family religion, the religion, the religion of the old dispensation in China and the Church religion in all countries, by the institution of the sacrament and inviolability of marriage establishes the Family, so what I have called the State religion in China which Confucius taught, by the institution of this new sacrament of the contract of allegiance, establishes the State. If you will consider what a great service the man who first instituted the sacrament and established the inviolability of marriage in the world has done for humanity and the cause of civilisation, you will then, I think, understand what a great work this is which Confucius did when he instituted this new sacrament and established the inviolability of the contract of allegiance. The institution of the sacrament of marriage secures the stability and permanence of the Family, without which the human race would become extinct. The institution of this sacrament of the contract of allegiance secures the stability and permanence of the State, without which human society and civilisation would all be destroyed and mankind would return to the state of savages or animals. I therefore said to you that the greatest thing which Confucius has done for the Chinese people is that he gave them the true idea of a State -- a true, rational, permanent, and absolute basis of a State, and in giving them that, he made it a religion,  -- a State religion.

Confucius taught this State religion in a book which, as I told you, he wrote in the very last days of his life, a book to which he gave the name of Spring and Autumn. In this book Confucius first instituted the new sacrament of the contract of allegiance called \emph{ming fen ta yi}, or the Code of Honour. This sacrament is therefore often and generally spoken of as \emph{Chun Chiu ming fen ta yi} (春秋名分大義) or simply \emph{Chun Chiu ta yi} -- (春秋大義) i.e., the Great Principle of Honour and Duty of the Spring and Autumn Annals, or simply the Great Principle or Code of the Spring and Autumn Annals. This book in which Confucius taught the Divine duty of loyalty is the Magna Charta of the Chinese nation. It contains the sacred covenant, the sacred social contract by which Confucius bound the whole Chinese people and nation to be absolutely loyal to the Emperor, and this covenant or sacrament, this Code of Honour, is the one and only true Constitution not only of the State and Government in China, but also of the Chinese civilisation. Confucius said it is by this book that after ages would know him -- know what he had done for the world.

I am afraid I have exhausted your patience in taking such a very long way to come to the point of what I want to say. But now we have got to the point where I last left you. You will remember I said that the reason why the mass of mankind will always feel the need of religion -- I mean religion in the European sense of the word -- is because religion gives them a refuge, one refuge, the belief in an all powerful Being called God in which they can find a sense of permanence in their existence. But I said that the system of philosophy and morality which Confucius taught, known as Confucianism, can take the place of religion, can make men, even the mass of mankind do without religion. Therefore, there must be, I said, something in Confucianism which can give to men, to the mass of mankind, the same sense of security and sense of permanence which religion gives. Now, I think we have found this something. This something is the \emph{Divine duty of loyalty to the Emperor} taught by Confucius in the State religion which he has given to the Chinese nation.

Now, this absolute Divine duty of loyalty to the Emperor of every man, woman, and child in the whole Chinese Empire gives, as you can understand, in the minds of the Chinese population, an absolute, supreme, transcendent, almighty power to the Emperor; and this belief in the absolute, supreme, transcendent, almighty power of the Emperor it is which gives to the Chinese people, to the mass of the population in China, the same sense of security which the belief in God in religion gives to the mass of mankind in other countries. The belief in the absolute, supreme, transcendent, almighty power of the Emperor also secures in the minds of the Chinese population the absolute stability and permanence of the State. This absolute stability and permanence of the State again secures the infinite continuance and lastingness of society. This infinite continuance and lastingness of society finally secures in the minds of the Chinese population the immortality of the race. Thus it is this belief in the immortality of the race, derived from the belief in the almighty power of the Emperor given to him by the Divine duty of loyalty, which gives to the Chinese people, the mass of the population in China, the same sense of permanence in their existence which the belief in a future life of religion gives to the mass of mankind in other countries.

Again, as the absolute Divine duty of loyalty taught by Confucius secures the immortality of the race in the nation,
so the cult of ancestor-worship taught in Confucianism secures the immortality of the race in the family.
Indeed, the cult of ancestor-worship in China is not founded much on the belief in a future life as in the belief of the immortality of the race.
A Chinese, when he dies, is not consoled by the belief that he will live a life hereafter,
but by the belief that his children, grandchildren, great-grand-children, all those dearest to him,
will remember him, think of him, love him, to the end of time,
and in that way, in his imagination, dying, to a Chinese, is like going on a long, long journey,
if not with the hope, at least with a great "perhaps" of meeting again.
Thus this cult of ancestor-worship, together with the Divine duty of loyalty,
in Confucianism gives to the Chinese people the same sense of permanence in their existence while they live and the same consolation when they die
which the belief in a future life in religion gives to the mass of mankind in other countries.
It is for his reason that the Chinese people attach the same importance to this cult of ancestor-worship
as they do to the principle of the Divine duty of loyalty to the Emperor.
Mencius said: ``Of the three great sins against filial piety the greatest is to have no posterity."
Thus the whole system of teaching of Confucius which I have called the State religion in China consists really only of two things,
loyalty to the Emperor and filial piety to parents -- in Chinese, \emph{Chung Hsiao}(忠孝) .
Intact, the three Articles of Faith, called in Chinese the \emph{san kang}(三纲), three cardinal duties in Confucianism or the State religion of China, are, in their order of importance --
first, absolute duty of loyalty to the Emperor;
second, filial piety and ancestor-worship;
third, inviolability of marriage and absolute submission of the wife to the husband.
The last two of the three Articles were already in what I have called the Family religion,
or religion of the old dispensation in China before Confucius' time;
but the first Article -- absolute duty of loyalty to the Emperor
-- was first taught by Confucius and laid down by him in the State religion or religion of the new dispensation which he gave to the Chinese nation.
This first Article of Faith -- absolute duty of loyalty to the Emperor
-- in Confucianism takes the place and is the equivalent of the First Article of Faith in all religions
-- the belief in God.
It is because Confucianism has this equivalent for the belief in God of religion
that Confucianism,
as I have shown you, can take the place of religion,
and the Chinese people, even the mass of the population in China,
do not feel the need of religion.

But now you will ask me how without a belief in God which religion teaches, how can one make men, make the mass of mankind, follow and obey the moral rule which Confucius teaches, the absolute duty of loyalty to the Emperor, as you can by the authority of God which the belief in God gives, make men follow and obey moral rules given by religion? Before I answer your question, will you allow me first to point out to you a great mistake which people make in believing that it is the sanction given by the authority of God which makes men obey the rules of moral conduct. I told you that the sanction for the sacrament and inviolability of marriage in Europe is given by the
Church, and the authority for the sanction, the Church says, is from God. But I said that was only an outward formal sanction. The real true inner sanction for the inviolability of marriage as we see it in all countries where there is no Church religion, is the sense of honour, the law of the gentleman in the man and woman. Thus the real authority for the obligation to obey rules of moral conduct is the moral sense, the law of the gentleman, in man. The belief in God is, therefore, not necessary to make men obey rules of moral conduct.

It is this fact which has made sceptics like Voltaire and Tom Paine in the last century, and rationalists like Sir Hiram Maxim today, say,
that the belief in God is a fraud or imposture invented by the founders of religion and kept up by priests.
But that is a gross and preposterous libel. All great men, all men with great intellect, have all always believed in God.
Confucius also believed in God, although he seldom spoke of it. Even Napoleon with his great, practical intellect believed in God.
As the Psalmist says: ``Only the fool -- the man with a vulgar and shallow intellect -- has said in his heart, `There is no God. ' ''
But the belief in God of man of great intellect is different from the belief in God of the mass of mankind.
The belief in God of men of great intellect is that of Spinoza: a belief in the Divine Order of the Universe.
Confucius said: ``At fifty I knew the Ordinance of God'' \footnote{论语 -- Discourses and Sayings Chap. II 4.}
 -- i.e., the Divine Order of the Universe.
 Men of great intellect have given different names to this Divine Order of the Universe.
 The German Fichte calls it the Divine idea of the Universe. In philosophical language in China it is called \emph{Tao} -- the Way.
But whatever name men of great intellect may give to this Divine Order of the Universe,
it is the knowledge of this Divine Order of the Universe which makes men of great intellect see the \emph{absolute} necessity of obeying rules of moral conduct or moral laws
which form part of that Divine Order of the Universe.

Thus, although the belief in God is not necessary to make men obey the rules of moral conduct, yet the belief in God is necessary to make men see the \emph{absolute} necessity of obeying these rules.
It is the knowledge of the absolute necessity of obeying the rules of moral conduct which enables and makes all men of great intellect follow and obey those rules.
Confucius says: ``A man without a knowledge of the Ordinance of God, i.e., the Divine Order of the Universe, will not be able to be a gentleman or moral man.'' \footnote{Discourses and Sayings Chap. XX 3.}
But then, the mass of mankind, who have not great intellect, cannot follow the reasoning which leads men of great intellect to the knowledge of the Divine Order of the Universe and cannot therefore understand the absolute necessity of obeying moral laws. Indeed, as Matthew Arnold says:
``Moral rules, apprehended as ideas first, and then rigorously followed as laws are and must be for the sage only. The mass of mankind have neither force of intellect enough to apprehend them as ideas nor force of character enough to follow them strictly as laws. ''
It is for this reason that the philosophy and morality taught by Plato, Aristotle and Herbert Spencer have a value only for scholars.

But the value of religion is that it enables men, enables and can make even the mass of mankind who have not force of intellect nor force of character, to strictly follow and obey the rules of moral conduct . But then how and by what means does religion enable and make men do this? People imagine that religion enables and makes men obey the rules of moral conduct by teaching men the belief in God. But that, as I have shown you, is a great mistake. The one and sole authority which makes men really obey moral laws or rules of moral conduct is the moral sense, the law of the gentleman in them. Confucius said: "A moral law which is outside of man is not a moral law.

Even Christ in teaching His religion says: ``The Kingdom of God is within you.''
I say, therefore, the idea which people have that religion makes men obey the rules of moral conduct by means of teaching them the belief in God is a mistake. Martin Luther says admirably in his commentary on the Book of Daniel: ``A God is simply that where-on the human heart \emph{rests} with trust, faith, hope and love. If the resting is right, then the God, too, is right; if the resting is wrong, then the God, too, is illusory.
'' This belief in God taught by religion is, therefore, only a \emph{resting}, or, as I call it, a refuge.
But then Luther says: ``The resting, i.e. the belief in God, must be true, otherwise the resting, the belief, is illusory.
In other words, the belief in God must be a true knowledge of God, a real knowledge of the Divine Order of the Universe, which, as we know, only men of great intellect can attain and which the mass of mankind cannot attain.
Thus you see the belief in God taught by religion, which people imagine enables the mass of mankind to follow and obey the rules of moral conduct, is illusory.
Men rightly call this belief in God -- in the Divine Order of the Universe taught by religion -- a faith, a trust, or, as I called it, a refuge.
Nevertheless, this refuge, the belief in God, taught by religion, although illusory, an illusion, helps towards enabling men to obey the rules of moral conduct, for, as I said, the belief in God gives to men, to the mass of mankind, a sense of security and a sense of permanence in their existence.
Goethe says: ``Piety, (From-migkeit) i.e., the belief in God, taught by religion, is not an end in itself but only a means by which, through the complete and perfect calmness of mind and temper (Gemuethsruehe) which it gives, to attain the highest state of culture or human perfection.''
In other words, the belief in God taught by religion, by giving men a sense of security and a sense of permanence in their existence, calms them, gives them the necessary calmness of mind and temper to feel the law of the gentleman or moral sense in them, which, I say again, is the one and sole authority to make men really obey the rules of moral conduct or moral laws.

But if the belief in God taught by religion only helps to make men obey the rules of moral conduct, what is it then upon which Religion depends principally to make men, to make the mass of mankind, obey the rules of moral conduct? It is \emph{inspiration}.
Matthew Arnold truly says: ``The noblest souls of whatever creed, the pagan Empedocles as well as the Christian Paul, have insisted on the necessity of inspiration, a living emotion to make moral actions perfect.''
Now what is this inspiration or living emotion in Religion, the paramount virtue of Religion upon which, as I said. Religion principally depends to make men, to enable and make even the mass of mankind obey the rules of moral conduct or moral laws?

You will remember I told you that the whole system of the teachings of Confucius may be summed up in one word; the Law of the Gentleman, the nearest equivalent for which in the European languages, I said, is moral law.
Confucius calls this law of the gentleman a secret.\footnote{中庸 -- The Universal order XII 1.} 
Confucius says: ``The law of the gentleman is to be found everywhere, and yet it is a secret.''
Nevertheless Confucius says: ``The simple intelligence of ordinary men and women of the people even can know something of this secret.
The ignoble nature of ordinary men and women of the people, too, can carry out this law of the gentleman.''
For this reason Goethe, who also knew this secret -- the law of the gentleman of Confucius, called it an "open secret.
''Now where and how did mankind come to discover this secret?
Confucuis said, you will remember, I told you that the recognition of the law of the gentleman began with the recognition of the relation of husband and wife -- the true relation between a man and woman in marriage.
Thus the secret, the open secret of Goethe, the law of the gentleman of Confucius, was first discovered by a man and woman.
But now, a-gain, how did the man and the woman discover this secret -- the law of the gentleman of Confucius?

I told you that the nearest equivalent in the European languages for the law of the gentleman of Confucius, is moral law.
Now what is the difference between the law of the gentleman of Confucius and moral law
-- I mean the moral law or law of morality of the philosopher and moralist as distinguished from religion or law of morality taught by religious teachers.
In order to understand this difference between the law of the gentleman of Confucius and the moral law of the philosopher and moralist, let us first find out the difference that there is between religion and the moral law of the philosopher and moralist.
Confucius says: ``The Ordinance of God is what we call the law of our being.
To fulfil the law of our being is what we call the Moral Law.
The Moral Law when refined and put into proper order is what we call Religion.''\footnote{中庸 The Universal Order I.1.}
Thus, according to Confucius, the difference between Religion and moral law
-- the moral law of the philosopher and moralist
-- is that Religion is a refined and well ordered moral law, a deeper or higher standard of moral law.

The moral law of the philosopher tells us we must obey the law of our being called Reason.
But Reason, as it is generally understood, means our reasoning power, that slow process of mind or intellect which enables us to distinguish and recognise the definable properties and qualities of the outward forms of  --  things.
Reason, our reasoning power, therefore, enables us to see in moral relations only the definable properties and qualities, the \emph{mores}, the morality, as it is rightly called, the outward manner and dead form, the body, so to speak, of right and wrong, or justice.
Reason, our reasoning power alone, cannot make us see the undefinable, living, absolute essence of right and wrong, or justice, the life or soul, so to speak, of justice.
For this reason Laotzu says: ``The moral law that can be expressed in language is not the absolute moral law.
The moral idea that can be defined with words is not the absolute moral idea.'' \footnote{道可道非常道名可名非常名}
The moral law of the moralist again tells us we must obey the law of our being, called Conscience, i.e., our heart.
But then, as the Wise Man in the Hebrew Bible says, there are many devices in a man's heart.
Therefore, when we take Conscience, our heart, as the law of our being and obey it,
we are liable and apt to obey, not the voice of what I have called the soul of justice,
the indefinable absolute essence of justice, but the many devices in a man's heart.

In other words Religion tells us in obeying the law of our being we must obey the true law of our being,
not the animal or carnal law of our being called by St. Paul the \emph{law of the mind of the flesh},
and very well defined by the famous disciple of Auguste Comte, Monsieur Littre, as the law of self preservation and reproduction; but the true law of our being called by St. Paul the \emph{law of the mind of the Spirit},
and defined by Confucius as the law of the gentleman.
In short, this true law of our being, which Religion tells us to obey,
is what Christ calls the Kingdom of God within us.
Thus we see, as Confucius says. Religion is a refined, spiritualized, well-ordered moral law,
a deeper higher standard of moral law than the moral law of the philosopher and moralist.
Therefore, Christ said:
``Except your righteousness (or morality) exceed the righteousness (or morality) of the Scribes and Pharisees (ie., philosopher and moralist) ye shall in no wise enter into the Kingdom of Heaven.''

Now, like Religion, the law of the gentleman of Confucius is also a refined, well-ordered moral law
-- a deeper higher standard of moral law than the moral law of the philosopher and moralist.
The moral law of the philosopher and moralist tells us we must obey the law of our being called by the philosopher, Reason, and by the moralist, Conscience.
But, like Religion, the law of the gentleman of Confucius tells us we must obey the \emph{true} law of our being,
not the law of being of the average man in the street or of the vulgar and impure person,
but the law of being of what Emerson calls ``the simplest and purest minds'' in the world.
In fact, in order to know what the law of being of the gentleman is, we must first \emph{be a gentleman} and have,
in the words of Emerson, the simple and pure mind of the gentleman developed in him.
For this reason Confucius says: ``It is the man that can raise the standard of the moral law, and not the moral law that can raise the standard of the man.'' \footnote{论语 -- Discourses and Sayings Chap. XV 28.}

Nevertheless Confucius says we can know what the law of the gentleman is, if we will study and try to acquire the fine feeling or \emph{good taste} of the gentleman.
The word in Chinese \emph{li} (礼) for good taste in the teaching of Confucius has been variously translated as ceremony, propriety, and good manners, but the word means really \emph{good taste}.
Now? this good taste, the fine feeling and good taste of a gentleman, when applied to moral action, is what, in European language, is called the sense of honour.
In fact, the law of the gentleman of Confucius is nothing else but the sense of honour.
This sense of honour, called by Confucius the law of the gentleman, is not like the moral law of the philosopher and moralist, a dry, dead knowledge of the form or formula of right and wrong, but like the Righteousness of the Bible in Christianity, an instinctive, living, vivid perception of the indefinable, absolute essence of right and wrong or justice, the life and soul of justice called Honour.

Now, we can answer the question: How did the man and woman who first recognised the relation of husband and wife, discover the secret, the secret of Goethe, the law of the gentleman of Confucius?
The man and woman who discovered this secret, discovered it because they had the fine feeling, the good taste of the gentleman, called when applied to moral action the sense of honour, which made them see the undefinable, absolute essence of right and wrong or justice, the life and soul of justice called Honour.
But then what gave, what inspired the man and woman to have this fine feeling, this good taste or sense of honour which made them see the soul of justice called Honour?
This beautiful sentence of Joubert will explain it.
Joubert says: ``Les hommes no sont justes qu'envers ceux qu'ils aiment.
Man cannot be truly just to his neighbour unless he \emph{loves} him''.
Therefore the inspiration which made the man and woman see what Joubert calls true justice, the soul of justice called Honour,
and thus enable them to discover the secret -- the open secret of Goethe, the law of the gentleman of Coufucius  -- is Love -- the love between the man and the woman which gave birth, so to speak, to the law of the gentleman;
that secret, the possession of which has enabled mankind not only to build up society and civilisation, but also to establish religion -- to find God.
You can now understand Goethe's confession of faith which he puts into the mouth of Faust, beginning with the words:
\begin{quote}
Lifts not the Heaven its dome above? \\
Doth not the firm-set Earth beneath us lie?
\end{quote}

Now, I told you that it is not the belief in God taught by religion, which makes men obey the rules of moral conduct. What really makes men obey the rules of moral conduct is the law of the gentleman
-- the Kingdom of Heaven within us
-- to which religion appeals.
Therefore the law of the gentleman is really the life of religion, whereas the belief in God together with the rules of moral conduct which religion teaches,
is only the body, so to speak, of religion.
But if the life of religion is the law of the gentleman, the \emph{soul} of religion, the source of inspiration in religion,  -- is Love.
This love does not merely mean the love between a man and a woman from whom mankind only first learn to know it.
Love includes all true human affection, the feelings of affection between parents and children as well as the emotion of love and kindness, pity, compassion, mercy towards all creatures;
in fact, all true human emotions contained in that Chinese word \emph{Jen}(仁), for which the nearest equivalent in the European languages is, in the old dialect of Christianity, godliness, because it is the most godlike quality in man, and in modern dialect, humanity, love of humanity, or, in one word, love.
In short, the soul of religion, the source of inspiration in religion is this Chinese word \emph{Jen}, love -- or call it by what name you like -- which first came into the world as love between a man and a woman.
This, then, is the inspiration in religion, the paramount virtue in religion, upon which religion, as I said, depends principally to make men, to enable and make even the mass of mankind obey the rules of moral conduct or moral laws which form part of the Divine Order of the universe.
Confucius says: ``The law of the gentleman begins with the recognition of husband and wife; but in its utmost reaches, it reigns and rules supreme over heaven and earth -- the whole universe.''

We have now found the inspiration, the living emotion that is in religion. But this inspiration or living emotion in religion is found not only in religion -- I mean Church religion.
This inspiration or living e-motion is known to everyone who has ever felt an impulse which makes him obey the rules of moral conduct above all considerations of self-interest or fear.
In fact, this inspiration or living emotion that is in religion is found in every action of men which is not prompted by the base motive of self-interest or fear, but by the sense of duty and honour.
This inspiration or living emotion in religion, I say, is found not only in religion.
But the value of religion is that the words of the rules of moral conduct which the founders of all great religions have left behind them have, what the rules of morality of philosophers and moralists have not, this inspiration or living emotion which, as Matthew Arnold says, lights up those rules and makes it easy for men to obey them. But this inspiration or living emotion in the words of the rules of conduct of religion again is found not only in religion.
All the words of really great men in literature, especially poets, have also this inspiration or living emotion that is in religion. The words of Goethe, for instance, which I have just quoted, have also this inspiration or living emotion. But the words of great men in literature, unfortunately, cannot reach the mass of mankind because all great men in literature speak the language of educated men, which the mass of mankind cannot understand. The founders of all the great religions in the world have this advantage, that they were mostly uneducated men, and, speaking the simple language of uneducated men, can make the mass of mankind understand them.
The real value, therefore, of religion, the real value of all the great religions in the world, is that it can convey the inspiration or living emotion which it contains even to the mass of mankind.
In order to understand how this inspiration or living emotion came into religion, into all the great religions of the world, let us find out how these religions came into the world.

Now, the founders of all the great religions in the world, as we know, were all of them men of exceptionally or even abnormally strong emotional nature.
This abnormally strong emotional nature made them feel intensely the emotion of love or human affection, which, as I have said, is the source of the inspiration in religion, the soul of religion.
This intense feeling or emotion of love or human affection enabled them to see what I have called the indefinable, absolute essence of right and wrong or justice, the soul of justice which they called righteousness, and this vivid perception of the absolute essence of justice enabled them to see the unity of the laws of right and wrong or moral laws.
As they were men of exceptionally strong e-motional nature, they had a powerful imagination, which unconsciously personified this unity of moral laws as an almighty supernatural Being.
To this supernatural almighty Being, the personified unity of moral laws of their imagination, they gave the name of God, from whom they also believed that the intense feeling or emotion of love or human affection, which they felt, came.
In this way, then, the inspiration or living emotion that is in religion came into religion; the inspiration that lights up the rules of moral conduct of religion and supplies the emotion or motive power needful for carrying the mass of mankind, along the straight and narrow way of moral conduct.
But now the value of religion is not only that it has an inspiration or living emotion in its rules of moral conduct which lights up these rules and makes it easy for men to obey them.
The value of religion, of all the great religions in the world, is that they have an organisation for awakening, exciting, and kindling the inspiration or living emotion in men necessary to make them obey the rules of moral conduct.
This organisation in all the great religions of the world is called the Church.

The Church, many people believe, is founded to teach men the belief in God.
But that is a great mistake. It is this great mistake of the Christian Churches in modern times which has made honest men like the late Mr. J.A. Froude feel disgusted with the modern Christian Churches.
Mr. Froude says:
``Many a hundred sermons have I heard in England on the mysteries of the faith, on the divine mission of the clergy, on apostolic succession, etc.,
but never one that I can recollect on common honesty, on those primitive commandments,
`Thou shalt not lie' and `Thou shalt not steal.' ''
But then, with all deference to Mr. Froude, I think he is also wrong when he says here that the Church, the Christian Church, ought to teach morality.
The aim of the establishment of the Church no doubt is to make men moral, to make men obey the rules of moral conduct such as ``Thou shalt not lie" and "Thou shalt not steal.''
But the function, the true function of the Church in all the great religions of the world, is not to teach morality, but to teach \emph{religion},
which, as I have shown you, is not a dead square rule such as ``Thou shalt not lie" and" Thou shalt not steal," but an inspiration, a living emotion to make men obey those rules.
The true function of the Church, therefore, is not to teach morality,
but to \emph{inspire} morality, to inspire men to be moral;
in fact, to inspire and fire men with a living emotion which makes them moral.
In other words, the Church in all the great religions of the world is an organisation, as I said, for awakening and kindling an inspiration or living emotion in men necessary to make them obey the rules of moral conduct.
But how does the Church awaken and kindle this inspiration in men?

Now, as we all know, the founders of all the great religions of the world not only gave an inspiration or living emotion to the rules of moral conduct which they taught, but they also inspired their immediate disciples with a feeling and emotion of unbounded admiration, love, and enthusiasm for their person and character.
When the great teachers died, their immediate disciples, in order to keep up the feeling and emotion of unbounded admiration, love, and enthusiasm which they felt for their teacher, founded a Church.
That, as we know, was the origin of the Church in all the great religions of the world.
The Church thus awakens and kindles the inspiration or living emotion in men necessary to make them obey the rules of moral conduct, by keeping up, exciting and arousing, the feeling and emotion of unbounded admiration, love, and enthusiasm for the person and character of the first Teacher and Founder of religion which the immediate disciples originally felt.
Men rightly call not only the belief in God, but the belief in religion a \emph{faith}, a trust; but a trust in whom? In the first teacher and founder of their religion who, in Mo-hammedanism is called the Prophet and in Christianity the Mediator.
If you ask a conscientious Mohammedan why he believes in God and obeys the rules of moral conduct, he will rightly answer you that he does it because he believes in Mohammed the Prophet.
If you ask a conscientious Christian why he believes in God and obeys the rules of moral conduct, he will rightly answer you that he does it because he \emph{loves} Christ.
Thus you see the belief in Mohammed, the love of Christ, in fact the feeling and emotion,
as I said of unbounded admiration, love, and enthusiasm for the first Teacher and Founder of religion
which it is the function of the Church to keep up, excite and arouse in men
-- is the source of inspiration, the real power in all the great religions of the world by which they are able to make men, to make the mass of mankind obey the rules of moral conduct.
\footnote{Mencius, speaking of the two purest and most Christlike characters in Chinese history, said: ``When men heard of the spirit and temper of Po-yi and Shu-ch'i, the dissolute ruffian became unselfish and the cowardly man had courage.'' Mencius Bk. III, Part II, IX, 11.}

I have been a long way, but now I can answer the question which you asked me awhile ago.
You asked me, you will remember, how without a belief in God which religion teaches
-- how can one make men, make the mass of mankind, follow and obey the moral rule which Confucius teaches in his State religion
-- the absolute duty of loyalty to the Emperor?
I have shown you that it is not the belief in God taught by religion which really makes men obey moral rules or rules of moral conduct.
I showed you that religion is able to make men obey the rules of moral conduct principally by means of an organisation called the Church which awakens and kindles in men an inspiration or living emotion necessary to make them to obey those rules.
Now, in answer to your question I am going to tell you that the system of the teachings of Confucius, called Confucianism,
the State religion in China, like the Church religion in other countries, makes men obey the rules of moral conduct also by means of an organisation corresponding to the Church of the Church religion in other countries. This organisation in the State religion of Confucianism in China is -- the \emph{school}. The school is the Church of the State religion of Confucius in China.
As you know, the same word ``\emph{chiao}'' in Chinese for religion is also the word for education. In fact, as the Church in China is the school, religion to the Chinese means education, culture.
The aim and object of the school in China is not, as in modern Europe and America to-day, to teach men how to earn a living, how to make money, but, like the aim and object of the Church religion, to teach men to understand what Mr. Froude calls the primitive commandment, ``Thou shalt not lie'' and ``Thou shall not steal'';
in fact, to teach men to be good.
``Whether we provide for action or conversation,'' says Dr. Johnson. ``whether we wish to be useful or pleasing, the first requisite is the religious and moral knowledge of right and wrong; the next, an acquaintance with the history of mankind and with those examples which may be said to embody truth and prove by events the reasonableness of opinions.''

But then we have seen that the Church of the Church religion is able to make men obey the rules of moral conduct by awakening and kindling in men an inspiration or living emotion, and that it awakens and kindles this inspiration or living emotion principally by exciting and arousing the feeling and emotion of unbounded admiration, love, and enthusiasm for the character and person of the first Teacher and Founder of religion.
Now, here there is a difference between the school -- the Church of the State religion of Confucius in China -- and the Church of the Church religion in other countries.
The school --  the Church of the State religion in China -- it is true, enables and makes men obey the rules of moral conduct, also like the Church of the Church religion, by awakening and kindling in men an inspiration or living emotion.
But the means which the school in China uses to awaken and kindle this inspiration or living emotion in men are different from those of the Church of the Church religion in other countries.
The school, the Church of the State religion of Confucius in China, does not awaken and kindle this inspiration or living emotion in men by exciting and arousing the feeling of unbounded admiration, love, and enthusiasm for Confucius.
Confucius in his lifetime did indeed inspire in his immediate disciples a feeling and emotion of unbounded admiration, love, and enthusiasm, and, after his death, has inspired the same feeling and emotion in all great men who have studied and understood him.
But Confucius even while he lived did not inspire, and, after his death, has not inspired in the mass of mankind the same feeling and emotion of admiration, love, and enthusiasm which the founders of all the great religions in the world, as we know, have inspired.
The mass of the population in China do not adore and worship Confucius as the mass of the population in Mohammedan countries adore and worship Mohammed, or as the mass of the population in European countries adore and worship Jesus Christ.
In this respect Confucius does not belong to the class of men called founders of a religion.
In order to be a founder of a religion in the European sense of the word, a man must have an exceptionally or even an abnormally strong emotional nature.
Confucius indeed was descended from a race of kings, the house of Shang, the dynasty which ruled over China before the dynasty under which Confucius lived -- a race of men who had the strong emotional nature of the Hebrew people.
But Confucius himself lived under the dynasty of the House of Chow -- a race of men who had the fine intellectual nature of the Greeks, a race of whom the Duke of Chou, the founder, as I told you, of the pre-Confucian religion or religion of the old dispensation in China was a true representative.
Thus Confucius was, if I may use a comparison, a Hebrew by birth, with the strong emotional nature of the Hebrew race, who was trained in the best intellectual culture, who had all that which the best intellectual culture of the civilisation of the Greeks could give him.
In fact, like the great Goethe in modern Europe,
the great Goethe whom the people of Europe will one day recognise as the most perfect type of humanity,
the \emph{real European} which the civilisation of Europe has produced,
as the Chinese have acknowledged Confucius to be the most perfect type of humanity,
the \emph{real Chinaman}, which the Chinese civilisation has produced
-- like the great Goethe, I say, Confucius was too educated and cultured a man to belong to the class of men called founders of religion.
Indeed, even while he lived Confucius was not known to be what he was, except by his most intimate and immediate disciples.

The school in China, I say, the Church of the State religion of Confucius, does not awaken and kindle the inspiration or living emotion necessary to make men obey the rules of moral conduct by exciting and arousing the feeling and emotion of admiration, love, and enthusiasm for Confucius.
But then how does the school in China awaken and kindle the inspiration or living emotion necessary to make man obey the rules of moral conduct?
Confucius says: ``In education the feeling and emotion is aroused by the study of \emph{poetry};
the judgement is formed by the study of good taste and good manners;
the education of the character is completed by the study of music.''
The school --  the Church of the State religion in China -- awakens and kindles the inspiration or living emotion in men necessary to make them obey the rules of moral conduct by teaching them poetry -- in fact, the works of all really great men in literature, which, as I told you, has the inspiration or living emotion that is in the rules of moral conduct of religion.
Matthew Arnold, speaking of Homer and the quality of \emph{nobleness} in his poetry, says:
``The nobleness in the poetry of Homer and of the few great men in literature can refine the raw, natural man, can \emph{transmute} him.''
In fact, whatsoever things are true, whatsoever things are just,
whatsoever things are pure, whatsoever things are lovely,
whatsoever things are of good report,
if there be any virtue and if there be any praise
-- the school, the Church of the State religion in China,
makes men think on these things, and in making them think on these things,
awakens and kindles the inspiration or living e-motion necessary
to enable and make them obey the rules of moral conduct.

But then you will remember I told you that the works of really great men in literature, such as the poetry of Homer, cannot reach the mass of mankind, because all great men in literature speak the language of educated men which the mass of mankind cannot understand.
Such being the case, how then does the system of the teachings of Confucius, Confucianism, the State Religion in China, awaken and kindle in the mass of mankind, in the mass of the population in China, the inspiration or living emotion necessary to enable and make them obey the rules of moral conduct? Now, I told you that the organisation in the State Religion of Confucius in China corresponding to the Church of the Church Religion in other countries, is the School.
But that is not quite correct.
The real organisation in the State Religion of Confucius in China corresponding exactly to the Church of the Church Religion in other countries is -- the \emph{Family}.
The real Church -- of which the School is but an adjunct
-- the real and true Church of the State Religion of Confucius in China,
is the Family with its ancestral tablet or chapel in every house,
and its ancestral Hall or Temple in every village and town.
I have shown you that the source of inspiration,
the real motive power by which all the great Religions of the world are able to make men,
to make the mass of mankind obey the rules of moral conduct,
is the feeling and emotion of unbounded admiration,
love and enthusiasm which it is the function of the Church to excite and arouse in men for the first Teachers and Founders of those Religions.
Now the source of inspiration, the real motive power by which the State Religion of Confucius in China is able to make men, to enable and make the mass of the population in China obey the rules of moral conduct is the " Love for their father and mother." The Church of the Church Religion, Christianity, says:
``Love Christ.''
The Church of the State Religion of Confucius in China
-- the ancestral tablet in every family
-- says ``Love your father and your mother."
St. Paul says: -- ``Let every man that names the name of Christ depart from iniquity. ''
But the author of the book on Filial Piety(孝经), written in the Han dynasty,
the counterpart of the \emph{Imitatio Christi} in China, says:
``Let everyone who loves his father and mother depart from iniquity.''
In short, as the essence, the motive power,
the source of real inspiration of the Church religion, Christianity,
is the Love of Christ, so the essence, the motive power,
the source of real inspiration of the State Religion,
Confucianism in China, is the ``Love of father and mother''
--  Filial Piety, with its cult of ancestor worship.

Confucius says: ``To gather in the same place where our fathers before us have gathered;
to perform the same ceremonies which they before us have performed;
to play the same music which they before us have played:
to pay respect to those whom they honoured;
to love those who were dear to them;
in fact, to serve them now dead as if they were living,
and now departed, as if they were still with us, that is the highest achievement of Filial Piety.''
Confucius, further says: -- ``By cultivating respect for the dead, and carrying the memory back to the distant past, the good in the people will grow deep.''
\emph{Cogitavi dies antiques, et annos eternos in menti habui}.
That is how the State Religion in China, Confucianism, awakens and kindles in men,
the inspiration or living emotion necessary to enable and make them obey the rules of moral conduct,
the highest and most important of all these rules being the absolute Duty of Loyalty to the Emperor,
just as the highest and most important rules of moral conduct in all the Great Religions of the world is fear of God.
In other words, the Church Religion, Christianity, says: -- ``Fear God and obey Him.''
But the State Religion of Confucius, or Confucianism, says: -- ``Honour the Emperor and be loyal to him.''
The Church Religion, Christianity, says: -- ``If you want to fear God and obey Him, you must first love Christ.''
The State Religion of Confucius, or Confucianism, say: -- ``If you want to honour the Emperor and be loyal to him, you must first love your father and mother.''

Now I have shown you why it is that there is no conflict between the heart and the head in the Chinese civilisation for these last 2,500 years since Confucius' time.
The reason why there is no such conflict is because the Chinese people, even the mass of the population in China, do not feel the need of Religion
-- I mean Religion in the European sense of the word; and the reason why the Chinese people do not feel the need of religion is because the Chinese people have in Confucianism something which can take the place of Religion.
That something, I have shown you, is the principle of absolute Duty of Loyalty to the Emperor; the Code of Honour called \emph{Ming fen ta yi},
which Confucius teaches in the State Religion which he has given to the Chinese nation.
The greatest service, I said, which Confucius has done for the Chinese people is in giving them this State Religion in which he taught the absolute Duty of Loyalty to the Emperor.

Thus much I have thought it necessary to say about Confucius and what he has done for the Chinese nation, because it has a very important bearing upon the subject of our present discussion,
the Spirit of the Chinese People.
For I want to tell you and you will understand it from what I have told you, that a Chinaman,
especially if he is an educated man, who knowingly forgets, gives up or throws away the Code of Honour,
the \emph{Ming fen ta yi} in the State Religion of Confucius in China,
which teaches the absolute Divine Duty of Loyalty to the Emperor or Sovereign to whom he has once given his alle-giance,
such a Chinaman is a man who has lost the spirit of the Chinese people,
the spirit of his nation and race: \emph{he is no longer a real Chinaman}.

Finally, let me shortly sum up what I want to say on the subject of our present discussion
-- the Spirit of the Chinese People or what is the real Chinaman.
The real Chinaman, I have shown you, is a man who lives the life of a man of adult reason with the simple heart of a child,
and the Spirit of the Chinese people is a happy union of soul with intellect.
Now if you will examine the products of the Chinese mind in their standard works of art and literature,
you will find that it is this happy union of soul with the intellect
-- which makes them so satisfying and delightful.
What Matthew Arnold says of the poetry of Homer is true of all Chinese standard literature, that
``it has not only the power of profoundly touching that natural heart of humanity,
which it is the weakness of Voltaire that he cannot reach,
but can also address the understanding with all Voltaire' s admirable simplicity and rationality.''

Matthew Arnold calls the poetry of the best Greek poets the priestess of imaginative reason. Now the spirit of the Chinese people, as it is seen in the best specimens of the products of their art and literature, is really what Matthew Arnold calls imaginative reason. Matthew Arnold says: -- "The poetry of later Paganism lived by the senses and understanding: the poetry of medieval Christianity lived by the heart and imagination. But the main element of the modern spirit's life, of the modern European spirit to-day, is neither the senses and understanding, nor the heart and imagination, it is the imaginative reason."

Now if it is true what Matthew Arnold says here that the element by which the modern spirit of the people of Europe to-day, if it would live right
-- has to live, is imaginative reason, then you can see how valuable for the people of Europe this Spirit of the Chinese people is,
-- this spirit which Matthew Arnold calls imaginative reason.
How valuable it is, I say, and how important it is that you should study it,
try to understand it, love it, instead of ignoring, despising and trying to destroy it.

But now before I finally conclude, I want to give you a warning. I want to warn you that when you think of this Spirit of the Chinese People, which I have tried to explain to you, you should bear in mind that it is not a science, philosophy, theosophy, or any "ism, " like the theosophy or " ism" of Madame Blavatsky or Mrs. Besant. The Spirit of the Chinese People is not even what you would call a mentality --  an active working of the brain and mind. The Spirit of the Chinese People, I want to tell you, is a state of mind, a temper of the soul, which you cannot learn as you learn shorthand or Esperanto -- in short, a mood, or in the words of the poet, a serene and blessed mood.

Now last of all I want to ask your permission to recite to you a few lines of poetry from the most Chinese of the English poets, Wordsworth, which better than anything I have said or can say, will describe to you the serene and blessed mood which is the Spirit of the Chinese People. These few lines of the English poet will put before you in a way I cannot hope to do, that happy union of soul with intellect in the Chinese type of humanity, that serene and blessed mood which gives to the real Chinaman his inexpressible gentleness. Wordsworth in his lines on Tintern Abbey says: --
\begin{quote}
``... nor less, I trust \\
To them I may have owed another gift \\
Of aspect more sublime: that blessed mood \\
In which the burthen of the mystery, \\
In which the heavy and the weary weight \\
Of all this unintelligible world, \\
Is lightened: -- that serene and blessed mood \\
In which the affections gently lead us on,  -- \\
Until, the breath of this corporeal frame \\
And even motion of our human blood \\
Almost suspended, we are laid asleep \\
In body, and become a living soul: \\
While with an eye made quiet by the power \\
Of harmony, and the deep power of joy, \\
We see into the life of things.''
\end{quote}

The serene and blessed mood which enables us \emph{to see into the life of things}:
that is imaginative reason, that is the Spirit of the Chinese People.
