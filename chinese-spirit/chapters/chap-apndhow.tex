
\chapter{辜鴻銘的西文學習法}

%\chapterauthors{餘一彥
  %\chapteraffil{(affiliation or affiliations)}
%}
餘一彥

〔編者按:本文摘自兒童中西文化導讀通訊第一期,40-42頁〕 


辜鴻銘,精通九國的語言文化,國學造詣極深,曾獲贈博士學位13個之多。他的思想影響跨越20世紀的東西方,是一位學貫中西、文理兼通的學者,又是近代中學西漸史上的先驅人物。

辜鴻銘I0歲時就隨他的義父一一英人布朗踏上蘇格蘭的土地,被送到當地一所著名的中學,受極嚴格的英國文學訓練。課餘的時間,布朗就親自教辜鴻銘學習德文。布朗的教法略異於西方的傳統,倒像是中國的私墊。他要求辜鴻銘隨他一起背誦歌德的長詩《浮士德》。布朗告訴辜鴻銘:"在西方有神人,卻極少有聖人。神人生而知之,聖人學而知之。西方只有歌德是文聖,毛奇是武聖。要想把德文學好,就必須背熟歌德的名著《浮士德》。"他總是比比劃劃地邊表演邊朗誦,要求辜鴻銘模仿著他的動作背誦,始終說說笑笑,輕鬆有趣。辜鴻銘極想知道《浮士德》書裏講的是什麼,但布朗堅持不肯逐字逐句地講解。他說:"只求你說得熟,並不求你聽得懂。聽懂再背,心就亂了,反倒背不熟了。等你把《浮士德》倒背如流之時我再講給你聽吧!"半年多的工夫辜鴻銘便稀裏糊塗地把一部《浮士德》大致背了下來。

第二年布朗才開始給辜鴻銘講解《浮士德》。他認為越是晚講,瞭解就越深,因為經典名作不同於一般著作,任何人也不能夠一聽就懂。這段時間裏辜鴻銘並沒有停頓對《浮士德》的記誦,已經可謂"倒背如流"了。

學完《浮士德》,辜鴻銘開始學"莎士比亞"的戲劇。布朗為辜鴻銘定下了半月學一部戲劇的計劃。八個月之後,見辜鴻銘記誦領會奇快,計劃又改為半月學三部。這樣大約不到一年,辜鴻銘已經把"莎士比亞"的37部戲劇都記熟了。

布朗認為辜鴻銘的英文和德文水準已經超過了一般大學畢業的文學士,將來足可運用自如了。但辜鴻銘只學了詩和戲劇,尚未正式涉及散文。布朗安排辜鴻銘讀卡萊爾的歷史名著《法國革命》。辜鴻銘此次基本轉入自學,自己慢慢讀慢慢背,遇有不懂的詞句再去請教別人。但只讀了三天,辜鴻銘就哭了起來。布朗吃驚地問:"怎樣了"辜鴻銘回答說:"散文不如戲劇好背。"布朗又問辜鴻銘背誦的進度,發現他每天讀三頁,於是釋然:"你每天讀得太多了。背誦散文作品每天半頁到一頁就夠多了。背誦散文同樣是求熟不求快,快而不熟則等於沒學。"

辜鴻銘所在的中學課業本來是極繁重的,但由於辜鴻銘各科在布朗身邊都提前打下了基礎,整個學習的過程便顯得毫不費力。學校的功課既然順利,沒事的辜鴻銘便接著記誦卡萊爾的《法國革命》。他越讀越有興致,可是讀多了便無法背熟。若按布朗的要求慢慢來,又控制不了自己的好奇心。就這樣時快時慢地把卡萊爾的《法國革命》讀完了。後來辜鴻銘終於徵得義父的同意,可以隨便閱讀義父布朗家中的藏書了。有許多書,辜鴻銘並沒有打算背熟,但也在不經意間"過目成誦"了。

布朗對義子的寄望極高。他曾告訴辜鴻銘:"我若有你的聰明,甘願作一個學者,拯救人類;不作一個百萬富翁,造福自己。讓我告訴你,現在歐洲國家和美國都想侵略中國,但是歐洲各國和美國的學者卻多想學習中國。我希望你能夠學通中西,就是為了教你擔起強化中國,教化歐美的重任,能夠給人類指出一條光明的大道,讓人能過上真正是人的生活!"

依照布朗的計劃,辜鴻銘應該先在英國學文、史、哲學及社會學,然後再到德國學習科學。學成之後才可以回中國修習傳統文化。布朗當初確實沒有看錯,辜鴻銘十四歲時,學術造詣就己經非一般人所能比。他只用了短短四年的時間,不僅初步完成了布朗擬定的家庭教學計劃,而且基本上修完了所在中學的各門主要課程。布朗不禁暗自為義子的聰明而感到驕傲。辜鴻銘在學校裏初步掌握了拉丁文和希臘文,其他課程的成績也都很出色,已經可以申請畢業了。

大約在1872年春季,辜鴻銘正式入愛丁堡大學就讀。辜鴻銘在愛丁堡大學的專修科為英國文學,同時兼修拉丁文、希臘文、數學、形而上學、道德哲學、自然哲學、修辭學等科目。辜鴻銘在學習拉丁文、希臘文時又不知暗自哭了多少次。他立志遍讀愛丁堡大學圖書館所藏希臘、拉丁文的文、史、哲名著。剛開始時,讀多少頁便背誦多少頁,還沒覺出多麼困難;後來隨著閱讀量的逐漸增大,漸漸感到吃不消了。他要自己堅持,再堅持,一定要一路背誦下去。辜鴻銘晚年憶及此事時曾悅:"說也奇怪,一通百通,像一條機器線,一拉開到頭。"到後來,不僅希臘、拉丁文,即如法、俄、意各國的語言、文學,辜鴻銘也能做到一學就會,觸類旁通。據悅辜鴻銘回國後,除本國語言外,尚能操九種文字與人交流,則其基礎主要是在愛丁堡大學讀書時打下的。

《論語.季氏》有雲:"生而知之者,上也。學而知之者,次也。困而學之,又其次也。困而不學,民斯為下矣。"至於"困"字的意思,舊註謂"有所不通",錢穆先生解作"經歷困境",辜鴻銘則自謂"吃不消「,他晚年曾對人說:"其實我讀書時主要的還是堅持『困而學之』的方法。久而久之不難掌握學習藝術,達到'不亦悅乎'的境地。旁人只看見我學習得多,學習得快,他們不知道我是用眼淚換來的!有些人認為記憶好壞是天生的,不錯,人的記憶力確實有優劣之分,但是認為記憶力不能增加是錯誤的。人心愈用而愈靈!"辜鴻銘憶起讀書時的往事,不禁慨嘆道:"困而不學,民斯為下矣!"(兆文鈞《辜鴻銘先生對我講述的往事》)則當時人們多認為辜鴻銘的博學在於他的天賦聰明,辜鴻銘自己是不承認的。

1871年4月,辜鴻銘以優秀的成績通過了所有相關科目的考試,在英國文學方面的學位考試中又表現非凡,順利獲得了愛丁堡大學文學碩士學位。這一年辜鴻銘僅20歲。

辜鴻銘自萊比錫大學畢業後,又赴巴黎短期進修法文。布朗又為辜鴻銘聯繫入巴黎大學,意在讓他學一些法學和政治學。其實當時辜鴻銘只22歲即已遍學科學、文學、哲學,並熟諳各國語言,造詣確非一般中國留學生可比。辜鴻銘以極快的速度讀完了巴黎大學整學期的講義和參考書,除偶爾去學校上點感興趣的課以外,辜鴻銘每天都抽一點時間教他的女房東學希臘文。從剛開始教她學希臘文字母那天起,辜鴻銘就教她背誦幾句《伊利亞特》。他的女房東笑著說:"你的教法真新鮮,沒聽說過。"於是,辜鴻銘就把布朗教自己背誦《浮士德》和莎翁戲劇的經過講給她聽。她說:"好,我這樣學下去。"辜鴻銘稅:"等你背熟一本,你就要背兩本,擋都擋不住。"

辜鴻銘的女房東常常拿著《伊利亞特》來到他的房間,把學過的詩句背給他聽,請求他的指點。辜鴻銘的教法果然有效,他的女房東在希臘文方面進展神速。許多客人見辜鴻銘教她學希臘文的方法與眾不同,都大為驚訝。

辜鴻銘後來曾對晚清直隸布政使淩福彭說:"學英文最好像英國人教孩子一樣的學,他們從小都學會背誦兒歌,稍大一點就教背詩背聖經,像中國人教孩子背四書五經一樣。"從辜鴻銘教他的女房東學希臘文的過程中可以看出,背誦《伊利亞特》的要旨即在於創造了一種真實的誦讀感受,如在希臘國土受希臘純正的啟蒙教育一般。此法乍看強度大,難度亦大,其實則不然。若由字母而單詞,再簡單拼句,則學習者在心理上就產生學外國語言的隔閡情緒了。辜鴻銘還依此法教會了他的女房東簡易的拉丁文,也不過三兩個月的工夫而已。

辜鴻銘深厚的西方素養極得益於童年背誦《浮士德》、《莎士比亞》的經歷。他後來在北京大學教英詩時,有學生向他請教掌握西文的妙法,他答曰:"先背熟一部名家著作作根基。"辜鴻銘曾說:"今人讀英文十年,開目僅能閱報,伸紙僅能修函,皆由幼年讀一貓一狗之式教科書,是以終其身只有小成。"他主張"中國私墊教授法,以開蒙未久,即讀四書五經,尤須背誦如流水也。"

