%This is file foreward.tex
\begin{foreword}
\chapter*{Foreword}

The concept of a {\it frontier} is a commonplace metaphor in the physical sciences, as well in the history of exploration. Today, one of the most tangible and alluring of all such frontiers is represented by the surface of Mars. This is because of the literally phenomenal scientific progress that has resulted from the intensified robotic exploration of the Red Planet since 1996. In little more than a decade (1996--2007), scientific viewpoints have been altered more profoundly than in the previous 30$+$ years. Some would describe this radical alteration in thinking as a {\it scientific revolution}. A case for this perspective is made in a convincing fashion here in ``The Martian Surface: Composition, Mineralogy, and Physical Properties,'' edited by Jim Bell and written by Jim and nearly a hundred other colleagues who study Mars for a living. Indeed, since the dawn of the Space Age, now in its 50th year (1957--2007), thoughts have often drifted to the so-called ``martian frontier'', with an ever-changing and sometimes disappointing scientific appreciation of what it might offer. This book puts the emerging ``new Mars'' into a modern scientific context on the basis of an ensemble of up-to-date scientific hypotheses and viewpoints. It brings Mars back alive and promotes prospects for future scientific exploration that are certain to continue the revolution at hand.

The Mars that scientific exploration has come to witness today is vastly more dynamic and scientifically interesting than that which the Viking missions of the 1970s revealed. When the last full compilation of scientific thinking about Mars was captured in the early 1990s (The 1992 University of Arizona Press book Mars, edited by Hugh Kieffer, Bruce Jakosky, Conway Snyder, and Mildred Matthews), the planet was effectively viewed as a nearly static geological world with intriguing but enigmatic climate cycles and little prospect for what we describe today as ``habitability'' or ``biological potential''. In the post-Viking view of Mars, all the dynamics of the planet and its hydrologic cycles were relegated to the most distant past, with only lurking and ephemeral signatures in the geology and atmosphere visible today. While interesting as one variety of silicate planet, Mars was not viewed as a scientific ``holy grail'', with revolutionary potential. NASA's only plans post-Viking converged upon a mission initially described as the ``Mars Geosciences and Climatology Orbiter'' (MGCO), which was later renamed {\it Mars Observer} in the latter part of the 1980s. This comprehensive mission was to have investigated the martian ``system'' in a fashion more akin to an Earth Observing System (EOS) than any traditional planetary remote sensing mission, in order to understand what scientific steps were justifiable in the competitive scientific landscape of the time.

When {\it Mars Observer} failed in the early 90s, the development of a more agile and distributed approach to Mars exploration was put in place, resulting in the reconnaissance observations of the Mars Global Surveyor (MGS). MGS catalyzed the scientific revolution that began in 1996 when the ALH84001 meteorite shocked the scientific and public communities into the renewed possibilities of life, or at least of primitive biological activity, on Mars. The measurements of MGS, however, provided the framework for quantifying and understanding a ``new Mars''. This framework, and the scientific impact of MGS as our views of the surface of Mars evolved from relative unknowns to well-measured systems, is articulated here by the authors of this book. For example, in June 2000 Mike Malin and Ken Edgett rocked the scientific community when they presented evidence for geologically recent runoff of liquid water on Mars, even despite the current understanding of its stability. This explosive discovery was a first glimmer of the revolution that was at hand. In the words of Steven Jay Gould, the mainstream thinking of this exciting time had its equilibrium punctuated by revolutionary discoveries that allowed a new set of theories about the role of water and potentially life on Mars to take root. ``The Martian Surface: Composition, Mineralogy, and Physical Properties'' paints for the reader a first-hand impression of the impact of such discoveries on the web of geological, geochemical, and climatological processes that shape any planet's surface.

Perhaps most catalytic in the unfolding martian scientific revolution has been the interplay of measurements from the armada of reconnaissance-oriented orbiters (MGS, Mars Odyssey, and ESA's Mars Express) and landed exploration via the Mars Exploration Rovers, Spirit and Opportunity. Indeed, the authors of this missive bring to light, for the first time, the emerging view of Mars that has been gleaned from the ongoing ``voyages'' of the rovers. This new view challenges the old post-Viking thinking by bringing the role of water into focus in ways that were somewhat unimagined just 30 years ago. While Mars may appear to have been a static, forever desiccated world, the discoveries that the Spirit and Opportunity rovers have made in their surface reconnaissance of the geochemical systems accessible on Mars today have painted a far different picture. From the ongoing work of the twin rovers to the just-commencing surveys of the Mars Reconnaissance Orbiter (MRO), it now appears as if Mars is indeed a ``water planet'', or a least a silicate planet in which the impact of water has manifested itself in a broad variety of scales and signatures. Understanding the many roles water has played in the evolution of the surface of Mars and its relation ultimately to the habitability of the Red Planet is elegantly portrayed in this book. Yet there is so much more to be learned\ldots

Mars has become a tangible scientific frontier thanks to the integrated measurements, experiments, and syntheses of the past decade. Fitting the story together is fraught with challenges, but the colleagues who have contributed to this timely summary and review of the field manage to succeed in a dramatic fashion. Their concluding arguments present a case for continuing the scientific conquest of the martian frontier in this new era of NASA's Vision for Space Exploration (VSE). Indeed, thanks to the pioneering efforts of the women and men around the world who are exploring Mars robotically (many of whom are co-authors of the chapters in this book), the path toward human exploration of Mars has been clarified and even accelerated. Mars is indeed a compelling scientific frontier; via the scientific framework presented here, we are closer to being there ourselves!\vskip\baselineskip

\noindent NASA Chief Scientist for Mars\hfill \author{James B. Garvin}\par
\noindent Exploration, 2000--2005\par
\noindent Greenbelt, Maryland\par
\noindent April 15, 2007\par

\end{foreword}