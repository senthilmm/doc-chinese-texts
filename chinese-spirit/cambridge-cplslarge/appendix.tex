%This is file appendix.tex
\chapter*{Appendix}

\section{SI\ units%
\index{SI units}}
We use SI throughout with the Sommerfeld convention%
\index{Sommerfield convention}
\begin{equation}
{p}{B}=\mu _{0}({H}+{M})  
\end{equation}%
Engineers prefer the Kennelly convention%
\index{Kennelly convention}
\begin{equation}
{B}=\mu _{0}{H}+{J}  
\end{equation}%
Both are consistent, compatible SI units since ${J}\mathbf{=}\mu
_{0}{M}$
The international system is based on the five basic quantities mass (m)
length (l) time (t) current (i) and temperature ($\theta $) with
corresponding units of kilogram, metre, second, ampere and kelvin. Derived
units include the newton (N) = kg m s$^{-2}$, joule (J) = N m, coulomb (C) =
A s, volt (V) = J\ C$^{-1}$, tesla (T) = J A$^{-1}$ m$^{-2}$ = V s m$^{-2}$,
weber (Wb) = V s = T m$^{2}$ and hertz (Hz) = s$^{-1}$
Recognized multiples are in steps of $10^{\pm 3}$, but a few exceptions are
admited such as centimetre (cm =10$^{-2}$ m) and Angstrom (\AA\ = 10$^{-10}$
m). Multiples of the metre are fm (10$^{-15}$), pm (10$^{-12}$), nm (10$%
^{-9} $), $%
%TCIMACRO{\U{b5}}%
%BeginExpansion
{\mu}%
%EndExpansion
$m (10$^{-6}$), mm (10$^{-3}$) m (10$^{0}$) and km (10$^{3}$).
Flux density $B$ is measured in telsa (also mT$,%
%TCIMACRO{\U{b5}}%
%BeginExpansion
{\mu}%
%EndExpansion
$T). Magnetic moment is measured in A m$^{2}$ so the magnetization and the $%
H $- field are measured in A m$^{-1}$. From (2.62) it is seen that an
equivalent unit for magnetic moment is J T$^{-1}$, so magnetization can also
be expressed as J T$^{-1}$ m$^{-3}$. $\sigma $, the magnetic moment per unit
mass in J T$^{-1}$ kg$^{-1}$ or A m$^{2%
{ }}$kg$^{-1}$ is the magnetic quantity most often measured in practice
in a vibrating-sample or SQUID magnetometer. The quantity $\mu _{0}$ is
exactly $4\pi $ $10^{-7}$ T m A$^{-1}$ and $\epsilon _{0}$ is deduced from
the speed of light $c$ = 2.998 10$^{8}$ m s$^{-1}$using $c^{2}=1/(\mu
_{0}\epsilon _{0}).$
The SI system has two compelling advantages for magnetism: (i) it is
possible to check the dimensions of any expression by inspection and (ii)
the units are directly related to the practical units of electricity. It is
the system used for undergraduate education in science and engineering
worldwide. A quantitative understanding of physical phenomena requires a
good grasp of the magnitudes of physical quantities, Such understanding is
not fostered by confusing different unit systems. SI is the mother tongue of
science. It is sensible to master your mother tongue before tackling another
language.
\subsection{cgs units%
\index{units!cgs}}
Most of the primary literature on magnetism is still written using cgs
units, or a muddled mixture where large fields are quoted in tesla and small
ones in oersted, one a unit of $B$, the other a unit of $H$! Basic cgs units
are cm, g and s. The electromagnetic unit of current is equivalent to 10 A.
The electromagnetic unit of potential is equivalent to 10 nV. The
electromagnetic unit of magnetic dipole moment (emu) is equivalent to 10$%
^{-3}$ A m$^{2}$. Derived cgs units include the erg (10$^{-7}$ J) so that an
energy density of 1 J m$^{-3}$ is equivalent to 10 erg cm$^{-3}$.
The convention relating flux density and magnetization in cgs is 
\begin{equation}
{B}={H}+4\pi {M}  
\end{equation}%
where the flux density or induction ${B}$ is measured in gauss
(G) and field ${H}$ in oersted (Oe). Magnetic moment is usually
expresed as emu, and magnetization is therefore emu cm$^{-3}$, although $%
4\pi {M}$ is considerd a flux-density expression, frequently
quoted in kilogauss. The magnetic constant $%
%TCIMACRO{\U{b5}}%
%BeginExpansion
{\mu}%
%EndExpansion
_{0}$ is numerically equal to 1 G Oe$^{-1}$, but its general omission from
the equations makes it impossible to check their dimensions.
The most useful conversion factors between SI and cgs units in magnetism are:
