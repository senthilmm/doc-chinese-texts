In memory of Fisher Black,  without whose tremendous contribution to
both theory and practice, derivatives research and industry would
not have reached the current stage, and certainly, this book which
concentrates on a Black-Scholes environment would not have been started.

With derivatives you can have almost any payoff pattern you want. 
If you can draw it on paper, or describe it in words, someone  can
design a derivative that gives you that payoff.

These days we often come across such terms as exotic options in
newspapers, journals, magazines, and many other financial reports.
You may wonder, as I did two years ago, what exactly they are. At
that time I had just started to work in the financial industry and
was much puzzled by the phrase ``exotic options''. Such puzzlement
left me feeling uneasy as I had  previously spent a few years at
school studying option pricing theory. I tried to find some
systematic sources to reeducate myself, and it turned out to be
nearly impossible as there was no systematic source on this subject.
Two years later, although the situation has changed  somewhat, exotic
options still remain mysterious to many people. I have tried to keep
a systematic record on this subject, though initially it was not my
intention to write a book on exotic options.

My first research paper on this subject was on spread options nearly
two years ago. Since then,  many other papers have followed. The
writing process has been so wonderful that I would never have learned
so fast and thoroughly had I not written this book. This book records
the accumulation of my knowledge on exotic options, and I would like
to share my learning curve with all of you.

To some degree, exotic options are as old as vanilla options. The
earliest article on exotic options can be traced to an article titled
``Alternative Forms of Options'' by Snyder, published in the {\it
Financial Analysts}.

{\it Journal}. It was in 1969, four years earlier than
the establishment of the Chicago Board of Options Exchange (CBOE),
the first organized options exchange in the world, and four years
earlier than the birth of the seminal work of Black and Scholes, who
made the path-breaking contribution to derivatives industry. However,
exotic options became somewhat popular only from the late 1970s and
have experienced significant growth in the past decade or so. The
primary motivations driving the recent innovations of derivatives are
cost-reduction and special customer needs such as off-balance sheet
opportunities, tax considerations, and so on.

The study of exotic options is indispensable not only for their own
active and important trading but also because they provide easy and
efficient building blocks for other more complicated financial
derivatives. In order not to inundate many readers, I try to spend a
significant amount of time in almost every chapter on how to use the
pricing formulas and how to apply them in practice.

A series of events in the derivatives industry since 1994, Orange
County, Kidder Peabody, Procter \& Gamble, Gibson Greetings, Askin
Capital, and so on, has attracted a lot of attention in the financial
industry as well as among the general public. These events created
calls for transparency of special-purpose derivatives activities. One
of the objects of this book is to provide a convenient source of
information for exotic options and thus to improve the transparency
of the  market. I try to provide as complete a source as possible on
this subject. In each chapter~I try to introduce one type of exotic
options, what it can achieve, and how to price and use it.

We will concentrate on a Black-Scholes environment throughout this
book for the purpose of transparency and easy comparisons with vanilla
options, because the Black-Scholes model is best known. In Section~4.3
of Chapter~4, we provide a derivation of the Black-Scholes formula
using the method to solve the related partial differential
equation. And in Section~4.4 of Chapter~4, we provide an intuitive and
concise method to derive the Black-Scholes formula. The method shown
in Section~4.4 to derive the Black-Scholes formula is the same method
we use to price essentially all exotic options in this book.

It is well-known in physics that energy can neither be created nor
destroyed, it can  only be transferred from one form to another.
Risk, or more specifically, financial risk, is such a complicated
subject that many researchers and financial institutions have been
struggling to find ways to measure it. Yet intuition suggests that
risk, like energy, can neither be created nor destroyed: it is
inherent within the financial system.



