
\begin{appendix}

\chapter{Dislocation Core and Atomic Force} %Appendix A
 
Early development of dislocation theory, and related theoretical
treatment of metallic properties controlled by dislocations, focussed
most attention on the effects of long-range elastic fields. In the
present context (see also Vitek, 1995) mechanical properties were
frequently analyzed in terms of long-range dislocation-dislocation,
dislocation-point defects etc interactions. The attitude prevailing in
the late 1960s that dislocation cores were of but secondary importance
in the plastic deformation of metals was radically altered in the next
two decades. It becomes widely recognized thatn the outcome. It
becomes widely recognized thatn the outcome.  As emphasized in the
studies of Vitek and co-workers (Vitek, 1985) clear signatures of core
effects are to be found in deformation modes and slip geometry, strong
orientation and temperature dependences of the yield stress, and also
in anomalous temperature dependence of the yield and flaw stresses
(see also Duesbery and Richardson, 1991).
\begin{equation} 
U_z =-{Ae^2\over r}\left[{1-\left({1\over s}\right)
\left({r_0\over r}\right)^{s-1}}\right].
\end{equation}

Significant impetus for such atomistic modeling has been the marked
improvement in experimental techniques (see Appendix~1), such as high
resolution electron microscopy (HREM), that are capable of atomic
resolution. As emphasized in the studies of Vitek and co-workers
(Vitek, 1985) clear signatures of core effects are to be found in
deformation modes and slip geometry, strong orientation and
temperature dependences of the yield stress, and also in anomalous
temperature dependence of the yield and flaw stresses (see also
Duesbery and Richardson, 1991).

Significant impetus for such atomistic modeling has been the marked
improvement in experimental techniques (see Appendix~1.), such as high
Significant impetus for such atomistic modeling has been the marked
improvement in experimental techniques (see Appendix~1.), such as high
Significant impetus for such atomistic modeling has been the marked
improvement in experimental techniques (see Appendix~1.), such as high
resolution electron microscopy (HREM), that are capable of atomic
resolution.

\section{Stacking Faults}  %A.1
Rosenberg (1978) discussed how close-packed planes of hard spheres 
can be stacked to form, say, a fcc structure, Fig.~1.2 being reproduced 
from his account.

\begin{figure}%A.1
%\epsfxsize=0.8\hsize
\ArtWork{fig-2.eps}
\caption{Caption for Fig.~2.}
\end{figure}

The first and second layers can be in positions labelled A and B 
while the third layer can be placed above the C positions. The pattern 
continues as ABCABC$\ldots,$ the pattern repeating at every third layer.

A stacking fault in such an fcc structure occurs if this sequence 
gets disturbed, as in ABCBCABC$\ldots.$ Here a layer A is missing, while in 
the sequence ABCABAC$\ldots$ an extra A layer has been introduced. While 
stacking fault can, at least in principle,extend through the entire 
crystal, they usually occupy only a part of the plane. In this last 
case, of a stacking fault which terminates within the crystal, the 
configuration at the termination is referred to as a partial dislocation.

\chapter{Glissile and Sessile Dislocations}  %Appendix B

Dislocations that can move by pure slip are called glissile. 
Dislocations which cannot glide, but have to move by some form of mass 
transports are called sessile (Read, 1953).

In crystals, the dislocation core spreads to certain crystallographic
planes containing the dislocation line. If the core spreads into one
of such planes, the core is planar and is glissile. If the core
spreads into several non-parallel planes of the zone of the
dislocation line, it is non-planer and is sessile. In the former case
the dislocation moves easily in the plane of the core spreading, while
in the latter case, it moves only with difficulty (Vitek, 1992). A
Shockley partial is a partial dislocation, the Burgers vector of which
lies in the plane of the fault. Then, Shockley partials are
glissile. A Frank partial is a partial dislocation, the Burgers vector
of which is not parallel to the fault. Then, Frank partials are
sessile. In the former case the dislocation moves easily in the plane
of the core spreading.

\section{Concept of Fractals} %B.1
Over a decade or more,diverse scientists have recognized that many 
of the structures common in their experiments have a quite special 
kind of geometrical complexity. The pioneering work was that of 
Mandelbrot (1977, 1979, 1982, 1988) who drew attention to the 
particular geometrical properties of such things as the shore of 
continents, tree branches,or the surface of clouds. Mandelbrot used 
the word `fractal' for these complex shapes,in order to emphasize 
that they are to be characterized by a non-integer (fractal) 
dimensionality.

Our interest here is the fractal aspects of fractured surfaces. 
Mandelbrot {\it et~al.}  (1984) gave an elegant route for determining the 
fractal dimension D of the surface. Their work pointed to a 
correlation between toughness and D. Further studies were performed 
by Lung (1986), Pande (1987), Lung and Mu (1988) and Xie and Chen 
(1988). Bouchard {\it et~al.}  (1990) later reported their findings that for 
a varity of ruptures modes and materials the observed fractal 
dimensions were the same to within the error bars. Dauskardt {\it et~al.}  
(1990) reported a fractal dimension $D\cong 2.2$, which, when 
combined with the studies of Bouchard {\it et~al.}  (1990) may turn out to 
be a universal value (but see also below).

\begin{table} %B.1
\tbl{Caption for Table~B.1.}{%
\begin{tabular}{@{}l@{\qquad\quad}l@{\qquad\quad}l@{\qquad\quad}l@{}}\toprule
Title 		&	Year	&   Author\\\colrule
X-ray photons	&	1912	&   M. von Laue\\
Electrons	&	1927	&   C. Davisson and L. H. Germer\\
He atoms	&	1930	&   O. Stern\\
$\rm H_2$ 
molecules	&	1936	&   O. Stern\\
Neutrons	&	1936	&   D. P. Mitchell and P. N. Powers\\\botrule
\end{tabular}
}
\end{table}

Though it is therefore known that cracks in nature can have fractal 
character, at the time of writing it is still difficult to specify 
just how this fracture nature arises. For it is certainly true that 
the mechanisms leading to fracture are highly material dependent (see 
Liebowitz, 1984). This, we will discuss more in Sec.~4.3.

Progress has resulted from modelling the growth of a single, connected
crack. With the assumption of contral forces numerical simulations of
media with a breaking probability proportional to the elongation of
springs revealed that the cracks resulting are fractal (Louis {\it
et~al.}, 1986; Hinrichen {\it et~al.}, 1989). The fractal dimension of
such cracks appears to be sensitive to the type of external force
(e.g. uniaxial tension, shear, uniform dilatation) but since only
rather small cracks can be grown more precision is lacking. Herrmann
(1989) has considered therefore deterministic models.

\end{appendix}

